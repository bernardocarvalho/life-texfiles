%%&program=xelatex
%&encoding=UTF-8 Unicode
% SVN keywords
% $Author: bernardo $
% $Date: 2014-07-24 19:33:49 +0100 (Thu, 24 Jul 2014) $
% $Revision: 6559 $
% $URL: http://metis.ipfn.ist.utl.pt/svn/cdaq/Users/Bernardo/Aulas/LFEB/teXfiles/Planck/Planck.tex $
\documentclass[a4paper,12pt]{article}  % Comments after  % are ignored
%\usepackage{hyperref}                 % For creating hyperlinks in cross references
%
\usepackage{ifxetex}% for XELATEX, or PDFlatex
\usepackage{ifplatform} 
%
\ifxetex
	\usepackage{polyglossia} \setmainlanguage{portuges}
	\usepackage{fontspec}
	\ifwindows
		\setmainfont[Ligatures=TeX]{Garamond}
		\setsansfont[Ligatures=TeX]{Gill Sans MT}
		\setmonofont{Consolas}
%		\setmonofont[Scale=MatchLowercase]{Courier}
	\fi
	\iflinux
		\setmainfont[Ligatures=TeX]{Linux Libertine O}
		\setsansfont[Ligatures=TeX,Scale=MatchLowercase]{Linux Biolinum}
		\setmonofont[Scale=MatchLowercase]{Courier}
	\fi
	\ifmacosx
	% add settings
	% Use xelatex -no-shell ...
	\fi
	\usepackage{xcolor,graphicx} 
\else
	\usepackage[portuguese]{babel}
	%\usepackage[latin1]{inputenc}
	\usepackage[utf8]{inputenc}
	\usepackage[T1]{fontenc}
	\usepackage{graphics}                 % Packages to allow inclusion of graphics
	\usepackage{color}                    % For creating coloured text and background
\fi

\usepackage{enumitem}
\setlist{nolistsep}

\usepackage{amsmath,amssymb,amsfonts} % Typical maths resource packages
\usepackage[retainorgcmds]{IEEEtrantools}
\usepackage{caption}

\usepackage{tikz}
\usetikzlibrary{calc,arrows,decorations.pathmorphing,intersections}

\usepackage[font={small,sf},labelfont={bf},labelsep=endash]{caption}
\usepackage{sansmath}

\def\width{18}
\def\hauteur{11}

\oddsidemargin 0cm
\evensidemargin 0cm

\pagestyle{myheadings}         % Option to put page headers
                               % Needed \documentclass[a4paper,twoside]{article}
\markboth{{\small \it  Laboratório de Física Experimental Básica}}
{{\small\it IST - MEFT -LFEB - 1º Sem. 2014/2015} }

\addtolength{\hoffset}{-0.5cm}
\addtolength{\textwidth}{2.5cm}
\addtolength{\topmargin}{-1.5cm}
\addtolength{\textheight}{3cm}

%\textwidth 15.5cm
%\topmargin -1.5cm
\setlength{\parindent}{0pt}
\setlength{\parskip}{1ex  plus  0.5ex  minus  0.2ex}
%\parindent 0.5cm
%\textheight 25cm
%\parskip 1mm


% Math macros
\newcommand{\ud}{\,\mathrm{d}} 
\newcommand{\HRule}{\rule{\linewidth}{0.5mm}}

\author{Prof. Bernardo B. Carvalho} 

%%%%, Bernardo Brotas Carvalho\\bernardo.carvalho@tecnico.ulisboa.pt} 
\date{ Outubro 2014} 

\begin{document} 

%	\includegraphics[width=0.2\textwidth]{../logo-ist}%\\[1cm]  %%  Logo_IST_color

%	\HRule \\[0.5cm]
%	{ \huge \sf  \textsc{Construções Geométricas em Lentes Delgadas (aproximação paraxial)} }\\[0.4cm] % \bfseries 
%	{ \large \bfseries Determinação da constante de Planck.}\\

%	{ \large \bfseries Procedimento Experimental}\\
%	\HRule \\%[0.5cm]
%TURNO:________ GRUPO:________ DATA: ___ \\
%Número: _______ Nome: ________________ \\
%Número: _______ Nome: ________________ \\
{  \sf  Relatório do Efeito fotoeléctrico.} %[0.4cm] % \bfseries 
Turno:\underline{\makebox[0.7cm][l]{~}} Grupo:\underline{\makebox[0.7cm][l]{~}} Data:\underline{\makebox[2cm][l]{~}}\\
\noindent Número:~\underline{\makebox[2cm][r]{~}} Nome:~\underline{\makebox[10cm][r]{~}} \\
\noindent Número:~\underline{\makebox[2cm][r]{~}} Nome:~\underline{\makebox[10cm][r]{~}} \\
\noindent Número:~\underline{\makebox[2cm][r]{~}} Nome:~\underline{\makebox[10cm][r]{~}} 


\section{\sf Questões a responder ANTES da sessão de Laboratório:}
\begin{enumerate}
\item Descreva por palavras suas quais os objectivos do Trabalho que irá realizar na sessão de Laboratório (uma folha A4). 
Indique as expressões que irá utilizar para obter as grandezas experimentais, bem como as expressões para calcular as incertezas. Inclua esta parte também no Relatório. Este irá constituir o ÚNICO meio de consulta na Prova Individual.

\item Para uma gota de chuva de diâmetro $R = 250\,\mu m$\footnote{Para gotas muito maiores a Lei de Stokes deixa de ser válida, pois a força de atrito passa a ser proporcional ao \emph{quadrado} da velocidade.}, $\rho_{H_2 O} = 1000 \, kg/m^{3}$, calcule a sua velocidade limite, a partir da espressão (\ref{eq:vlimit3}). Calcule o tempo característico, $\tau=m/k\,\eta_{ar}$, semelhante ao tempo necessário para alcançar a velocidade limite.
\item Qual a carga que a Terra teria de ter para que uma gota de óleo de $R = 10\,\mu m$, carregada com uma carga elementar positiva flutuar na atmosfera?
\item Pode o campo elétrico provocado pelas placas da motagem utilizada no laboratório ser considerado \emph{constante}?
\end{enumerate}


\section{\sf Relatório}
\subsection{\sf Montagem Experimental}
Desenhe um diagrama da experiência marcando as posições de cada componente e a posição de cada risca. Inclua uma lista e legenda de Instrumentos.

\begin{center}
\framebox[18cm]{\rule{0pt}{6.5cm}}
\end{center}

\subsection{\sf Dados Experimentais}

 Execute as Medições e preencha a tabela seguinte:

\noindent  Distância Percorrida $=$~\underline{\makebox[1cm][r]{~}}(m), 

\begin{center}
	%\centering
	\begin{tabular}{|c|c|c|c|c|c|}
	\hline
	Gôta \#  & $U_{paragem}$ [V] & Tempo [s] & $Veloc.$  [m/s]& $\overline{Veloc.}$ [m/s]	& $\delta Veloc.$ [m/s]  \\
	\hline
	  &  &  & &  & \\ \cline{2-4}
	1 &  & & &  & \\ \cline{2-4}
	  &  &  & &  & \\  \hline
	  &  &  & &  & \\ \cline{2-4}
	2 &  & & &  & \\ \cline{2-4}
	  &  &  & &  & \\ \hline
	  &  &  & &  & \\ \cline{2-4}
	3 &  & & &  & \\ \cline{2-4}
	  &  &  & &  & \\ \hline
 	\end{tabular}
\end{center}

\subsubsection{\sf Cálculos}

%\vspace{1cm}
\begin{center}
	\renewcommand{\arraystretch}{1.5}
	\begin{tabular}{|c|c|c|c|c|c|c|}
	\hline
	Gôta \#  & $R$ [$m$] & $\delta R$  [$\mu m$] & $q$ [C] & $\delta q$ [C] & $q_{corrig.}$ [C] & $\delta q_{corrig.}$ [C] \\ \hline
	 1 & &  &  & &  &  \\ \hline
	 2 & &  &  & &  & \\ \hline
	 3 & &  &  & &  & \\ \hline
	 4 & &  &  & &  & \\ \hline
 	\end{tabular}
\end{center}

Para as gotas de carga mais baixa marque na escala abaixo os respectivos valores e as incertezas
\begin{center}
  \sansmath
    \begin{tikzpicture}[scale=0.5,font=\sffamily]
 	%axis
	\draw (0,0) -- coordinate (x axis mid) (30,0);
%    \draw (0,0) -- coordinate (y axis mid) (0,1.7);
	%labels      
	\node[below=0.8cm] at (x axis mid) {$q_{gota} [10^{-19} C] $ };
    	%ticks
    	\foreach \x in {1,3,...,17}
     		\draw (\x,1pt) -- (\x,-3pt) node[anchor=north] {\x};
    	\foreach \x in {0,2,...,16}
     		\draw (\x,1pt) -- (\x,-3pt);
    \end{tikzpicture}
\end{center}

 A partir dos resultados obtidos e atendendo
aos erros experimentais é possível ou não concluir sobre a quantificação da carga elétrica?

%\noindent  Desvio à Exatidão $=$~\underline{\makebox[1cm][r]{~}}(\%), 
%Incerteza relativa $=$~\underline{\makebox[1cm][r]{~}}($\%$) 

\subsection{\sf Análise, Conclusões e Comentários}
\noindent\underline{\makebox[\textwidth][r]{~}} \\
\noindent\underline{\makebox[\textwidth][r]{~}} \\
\noindent\underline{\makebox[\textwidth][r]{~}} \\
\noindent\underline{\makebox[\textwidth][r]{~}} \\
\noindent\underline{\makebox[\textwidth][r]{~}} \\
\noindent\underline{\makebox[\textwidth][r]{~}} \\
\noindent\underline{\makebox[\textwidth][r]{~}} \\
\noindent\underline{\makebox[\textwidth][r]{~}} \\
\noindent\underline{\makebox[\textwidth][r]{~}} \\
\noindent\underline{\makebox[\textwidth][r]{~}} \\
\noindent\underline{\makebox[\textwidth][r]{~}} \\
\noindent\underline{\makebox[\textwidth][r]{~}} \\
\noindent\underline{\makebox[\textwidth][r]{~}} \\
\noindent\underline{\makebox[\textwidth][r]{~}} \\
\noindent\underline{\makebox[\textwidth][r]{~}} \\




%\newpage

\end{document} 