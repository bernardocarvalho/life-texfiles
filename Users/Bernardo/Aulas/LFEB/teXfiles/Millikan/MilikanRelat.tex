%%&program=xelatex
%&encoding=UTF-8 Unicode
% SVN keywords
% $Author: bernardo $
% $Date: 2014-07-24 19:33:49 +0100 (Thu, 24 Jul 2014) $
% $Revision: 6559 $
% $URL: http://metis.ipfn.ist.utl.pt/svn/cdaq/Users/Bernardo/Aulas/LFEB/teXfiles/Planck/Planck.tex $
\documentclass[a4paper,12pt]{article}  % Comments after  % are ignored
%\usepackage{hyperref}                 % For creating hyperlinks in cross references
%
\usepackage{ifxetex}% for XELATEX, or PDFlatex
\usepackage{ifplatform} 
%
\ifxetex
	\usepackage{polyglossia} \setmainlanguage{portuges}
	\usepackage{fontspec}
	\ifwindows
		\setmainfont[Ligatures=TeX]{Garamond}
		\setsansfont[Ligatures=TeX]{Gill Sans MT}
		\setmonofont{Consolas}
%		\setmonofont[Scale=MatchLowercase]{Courier}
	\fi
	\iflinux
		\setmainfont[Ligatures=TeX]{Linux Libertine O}
		\setsansfont[Ligatures=TeX,Scale=MatchLowercase]{Linux Biolinum}
		\setmonofont[Scale=MatchLowercase]{Courier}
	\fi
	\ifmacosx
	% add settings
	% Use xelatex -no-shell ...
	\fi
	\usepackage{xcolor,graphicx} 
\else
	\usepackage[portuguese]{babel}
	%\usepackage[latin1]{inputenc}
	\usepackage[utf8]{inputenc}
	\usepackage[T1]{fontenc}
	\usepackage{graphics}                 % Packages to allow inclusion of graphics
	\usepackage{color}                    % For creating coloured text and background
\fi

\usepackage{enumitem}
\setlist{nolistsep}

\usepackage{amsmath,amssymb,amsfonts} % Typical maths resource packages
\usepackage[retainorgcmds]{IEEEtrantools}
\usepackage{caption}

\usepackage{tikz}
\usetikzlibrary{calc,arrows,decorations.pathmorphing,intersections}

\usepackage[font={small,sf},labelfont={bf},labelsep=endash]{caption}
\usepackage{sansmath}

\def\width{18}
\def\hauteur{11}

\oddsidemargin 0cm
\evensidemargin 0cm

\pagestyle{myheadings}         % Option to put page headers
                               % Needed \documentclass[a4paper,twoside]{article}
\markboth{{\small \it  Laboratório de Física Experimental Básica}}
{{\small\it IST - MEFT -LFEB - 1º Sem. 2015/2016} }

\addtolength{\hoffset}{-0.5cm}
\addtolength{\textwidth}{2.5cm}
\addtolength{\topmargin}{-1.5cm}
\addtolength{\textheight}{3cm}

%\textwidth 15.5cm
%\topmargin -1.5cm
\setlength{\parindent}{0pt}
\setlength{\parskip}{1ex  plus  0.5ex  minus  0.2ex}
%\parindent 0.5cm
%\textheight 25cm
%\parskip 1mm


% Math macros
\newcommand{\ud}{\,\mathrm{d}} 
\newcommand{\HRule}{\rule{\linewidth}{0.5mm}}

\author{Prof. Bernardo B. Carvalho} 

%%%%, Bernardo Brotas Carvalho\\bernardo.carvalho@tecnico.ulisboa.pt} 
\date{ Outubro 2014} 

\begin{document} 

%	\includegraphics[width=0.2\textwidth]{../logo-ist}%\\[1cm]  %%  Logo_IST_color

%	\HRule \\[0.5cm]
%	{ \huge \sf  \textsc{Construções Geométricas em Lentes Delgadas (aproximação paraxial)} }\\[0.4cm] % \bfseries 
%	{ \large \bfseries Determinação da constante de Planck.}\\

%	{ \large \bfseries Procedimento Experimental}\\
%	\HRule \\%[0.5cm]
%TURNO:________ GRUPO:________ DATA: ___ \\
%Número: _______ Nome: ________________ \\
%Número: _______ Nome: ________________ \\
{  \sf  Relatório da Exp. de Millikan.} %[0.4cm] % \bfseries 
Turno:\underline{\makebox[2cm][l]{~}} Grupo:\underline{\makebox[1cm][l]{~}} Data:\underline{\makebox[2cm][l]{~}}\\
\noindent Número:~\underline{\makebox[2cm][r]{~}} Nome:~\underline{\makebox[10cm][r]{~}} \\
\noindent Número:~\underline{\makebox[2cm][r]{~}} Nome:~\underline{\makebox[10cm][r]{~}} \\
\noindent Número:~\underline{\makebox[2cm][r]{~}} Nome:~\underline{\makebox[10cm][r]{~}} 


\section{\sf Trabalho preparatório a realizar  ANTES da sessão de Laboratório:}
\begin{enumerate}
\item Descreva por palavras suas quais os objectivos do Trabalho que irá realizar na sessão de Laboratório.
% (uma folha A4). 
%Indique as expressões que irá utilizar para obter as grandezas experimentais, bem como as expressões para calcular as incertezas. Inclua esta parte também no Relatório. Este irá constituir o ÚNICO meio de consulta na Prova Individual.

%\item Para uma gota de chuva de diâmetro $R = 250\,\mu m$\footnote{Para gotas muito maiores a Lei de Stokes deixa de ser válida, pois a força de atrito passa a ser proporcional ao \emph{quadrado} da velocidade.}, $\rho_{H_2 O} = 1000 \, kg/m^{3}$, calcule a sua velocidade limite, a partir da espressão (4) do Guia. 
%\ref{eq:vlimit3}
%Calcule o tempo característico, \\
%$\tau=m/k\,\eta_{ar}$, semelhante ao tempo necessário para alcançar a velocidade limite.
%\item Qual a carga que a Terra teria de ter para que uma gota de óleo de $R = 10\,\mu m$, carregada com uma carga elementar positiva flutuar na atmosfera?
%\item Nas condições  normais de temperatura e pressão (PTN) o volume molar do ar é $V_{mol}=22.41 l / mol$. Calcule a distância média das moléculas do ar e guarde o valor para discussão final sobre o ponto $2.3.2$ do Guia.
%$3.3 n m$ 
\end{enumerate}

\subsection{\sf Objectivos do Trabalho}
\noindent\underline{\makebox[\textwidth][r]{~}} \\
\noindent\underline{\makebox[\textwidth][r]{~}} \\
\noindent\underline{\makebox[\textwidth][r]{~}} \\
\noindent\underline{\makebox[\textwidth][r]{~}} \\
\noindent\underline{\makebox[\textwidth][r]{~}} \\
\noindent\underline{\makebox[\textwidth][r]{~}} \\
\noindent\underline{\makebox[\textwidth][r]{~}} \\

\subsubsection{\sf Equações }
Escreva no seguinte quadro todas as equações necessárias para calcular as grandezas bem com as suas incertezas.
\begin{center}
\framebox[15cm]{\rule{0pt}{6.5cm}}
\end{center}


\section{\sf Relatório}
\subsection{\sf Montagem Experimental}
Desenhe um diagrama da experiência, bem como um esboço da imagem que observa ao microscópio. Inclua uma lista com a Legenda de Instrumentos.

\begin{center}
\framebox[18cm]{\rule{0pt}{6.5cm}}
\end{center}

\subsection{\sf Dados Experimentais}\label{sec:dados}
Preencha as seguintes tabelas indicando  apenas os algarismos significativos. Poderá em alternativa utilizar folhas de cálculo, com o mesmo formato (apresentando-as em anexo) mas terá de peencher as colunas 2, 3, 5 e 6 da tabela seguintes e as colunas 6 e 7 das secção \ref{sec:calc}. Em qualquer dos casos terá que verificar as contas com auxílio calculadora, para um dos ensaios e na presença do docente.

 Execute as Medições e preencha a tabela seguinte :

\noindent  Distância Percorrida pelas gotas $=$~\underline{\makebox[1cm][r]{~}} div $=$~\underline{\makebox[1cm][r]{~}} m , 

\begin{center}
	%\centering
	\begin{tabular}{|c|c|c|c|c|c|}
	\hline
	Gota \#  & $U_{paragem}$ [V] & Tempo [s] & $Veloc.$  [m/s]& $\overline{Veloc.}$ [m/s]	& $\delta Veloc.$ [m/s]  \\
	\hline
	  &  &  & &  & \\ \cline{2-4}
	1 &  & & &  & \\ \cline{2-4}
	  &  &  & &  & \\  \hline
	  &  &  & &  & \\ \cline{2-4}
	2 &  & & &  & \\ \cline{2-4}
	  &  &  & &  & \\ \hline
	  &  &  & &  & \\ \cline{2-4}
	3 &  & & &  & \\ \cline{2-4}
	  &  &  & &  & \\ \hline
	  &  &  & &  & \\ \cline{2-4}
	4 &  & & &  & \\ \cline{2-4}
	  &  &  & &  & \\ \hline
 	\end{tabular}
\end{center}

\vspace{4cm}

\subsubsection{\sf Cálculo da carga das gotas}
\label{sec:calc}
\begin{center}
	\renewcommand{\arraystretch}{1.5}
	\begin{tabular}{|c|c|c|c|c|c|c|}
	\hline
	Gota \#  & $R$ [$\mu m$] & $\delta R$  [$\mu m$] & $q$ [C] & $\delta q$ [C] & $q_{corrig.}$ [C] & $\delta q_{corrig.}$ [C] \\ \hline
	 1 & &  &  & &  &  \\ \hline
	 2 & &  &  & &  & \\ \hline
	 3 & &  &  & &  & \\ \hline
	 4 & &  &  & &  & \\ \hline
 	\end{tabular}
\end{center}

\vspace{2cm}
No verso da folha marque numa escala horizontal a posição da carga das gotas de menor valor, as respectivas  margens de incertezas e múltiplos da menor carga,  como o exemplo abaixo indicado: 
\begin{center}
  \sansmath
    \begin{tikzpicture}[scale=0.5,font=\sffamily,
                    dot/.style={fill,circle,minimum size=1pt}]
 	%axis
	\draw[->] (0,0) -- coordinate (x axis right) (30,0);
%    \draw (0,0) -- coordinate (y axis mid) (0,1.7);
	%labels      
	\node[below=0.3cm] at (27.0,0)  {$q_{gota}\;\; \times 10^{-19} C $ };
    	%ticks
    	\foreach \x in {0,5,...,25}
     		\draw (\x,1pt) -- (\x,-3pt) node[anchor=north] {\x};
    	\foreach \x in {1,2,...,27}
     		\draw (\x,1pt) -- (\x,-3pt);
    	%\foreach \x in {0,2,...,26}
     	%	\draw (\x,1pt) -- (\x,-3pt);
    \draw[red, very thick] (1.3,0) -- (2.1,0);
    \draw[red,thick] (1.3,5pt) -- (1.3,-5pt);% node[anchor=north] {$q_1$};		
    \draw[red, very thick] (2.1,5pt) -- (2.1,-5pt);	
    \node[circle,fill,inner sep=1pt, red] (a) at (1.7,0) {};	
	\node[above=0.2cm, red] at (1.7,0) {$|q_{1}|$ };
	%
    \draw[green,  thick] (2*1.3,0) -- (2*2.1,0);
    \draw[green,thick] (2*1.3,5pt) -- (2*1.3,-5pt);% node[anchor=north] {$q_1$};		
    \draw[green,  thick] (2*2.1,5pt) -- (2*2.1,-5pt);	
    \node[circle,fill,inner sep=1pt, green] (a) at (2*1.7,0) {};	
	\node[above=0.2cm, green] at (2*1.7,0) {$2|q_{1}|$ };
	%
    \draw[red, very thick] (3.7,0) -- (5.9,0);
    \draw[red,thick] (3.7,5pt) -- (3.7,-5pt);		
    \draw[red, very thick] (5.9,5pt) -- (5.9,-5pt);		
    \node[circle,fill,inner sep=1pt, red] (a) at (4.8,0) {};	
	\node[above=0.2cm, red] at (4.8,0) {$|q_{2}|$ };
	%
 	%
    \draw[green,  thick] (6*1.3,0) -- (6*2.1,0);
    \draw[green,thick] (6*1.3,5pt) -- (6*1.3,-5pt);% node[anchor=north] {$q_1$};		
    \draw[green,  thick] (6*2.1,5pt) -- (6*2.1,-5pt);	
    \node[circle,fill,inner sep=1pt, green] (a) at (6*1.7,0) {};	
	\node[above=0.2cm, green] at (6*1.7,0) {$6| q_{1}|$ };

   \draw[red, very thick] (9.7,0) -- (16.3,0);
    \draw[red, very thick] (9.7,5pt) -- (9.7,-5pt);		
    \draw[red, very thick] (16.3,5pt) -- (16.3,-5pt);		
    \node[circle,fill,inner sep=1pt, red] (a) at (12.8,0) {};	
	\node[above=0.2cm, red] at (12.8,0) {$|q_{3}|$ };
    \end{tikzpicture}
\end{center}

Supondo que não conhecia o valor tabelado da carga do eletrão e apenas a partir dos resultados obtidos poderá tirar conclusões sobre a quantificação da carga elétrica?

%\noindent  Desvio à Exatidão $=$~\underline{\makebox[1cm][r]{~}}(\%), 
%Incerteza relativa $=$~\underline{\makebox[1cm][r]{~}}($\%$) 

\subsection{\sf Análise, Conclusões e Comentários}
\noindent\underline{\makebox[\textwidth][r]{~}} \\
\noindent\underline{\makebox[\textwidth][r]{~}} \\
\noindent\underline{\makebox[\textwidth][r]{~}} \\
\noindent\underline{\makebox[\textwidth][r]{~}} \\
\noindent\underline{\makebox[\textwidth][r]{~}} \\
\noindent\underline{\makebox[\textwidth][r]{~}} \\
\noindent\underline{\makebox[\textwidth][r]{~}} \\
\noindent\underline{\makebox[\textwidth][r]{~}} \\
\noindent\underline{\makebox[\textwidth][r]{~}} \\
\noindent\underline{\makebox[\textwidth][r]{~}} \\
\noindent\underline{\makebox[\textwidth][r]{~}} \\
\noindent\underline{\makebox[\textwidth][r]{~}} \\
\noindent\underline{\makebox[\textwidth][r]{~}} \\
\noindent\underline{\makebox[\textwidth][r]{~}} \\
\noindent\underline{\makebox[\textwidth][r]{~}} \\




%\newpage

\end{document} 