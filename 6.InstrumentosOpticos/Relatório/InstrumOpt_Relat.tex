%%&program=xelatex
%&encoding=UTF-8 Unicode
% SVN keywords
% $Author: bernardo $
% $Date: 2014-07-24 19:33:49 +0100 (Thu, 24 Jul 2014) $
% $Revision: 6559 $
% $URL: http://metis.ipfn.ist.utl.pt/svn/cdaq/Users/Bernardo/Aulas/LFEB/teXfiles/Planck/Planck.tex $
\documentclass[a4paper,12pt]{article}  % Comments after  % are ignored
%\usepackage{hyperref}                 % For creating hyperlinks in cross references
%
\usepackage{ifxetex}% for XELATEX, or PDFlatex
\usepackage{ifplatform} 
%
\ifxetex
	\usepackage{polyglossia} \setmainlanguage{portuges}
	\usepackage{fontspec}
	\ifwindows
		\setmainfont[Ligatures=TeX]{Garamond}
		\setsansfont[Ligatures=TeX]{Gill Sans MT}
		\setmonofont{Consolas}
%		\setmonofont[Scale=MatchLowercase]{Courier}
	\fi
	\iflinux
		\setmainfont[Ligatures=TeX]{Linux Libertine O}
		\setsansfont[Ligatures=TeX,Scale=MatchLowercase]{Linux Biolinum}
		\setmonofont[Scale=MatchLowercase]{Courier}
	\fi
	\ifmacosx
	% add settings
	% Use xelatex -no-shell ...
	\fi
	\usepackage{xcolor,graphicx} 
\else
	\usepackage[portuguese]{babel}
	%\usepackage[latin1]{inputenc}
	\usepackage[utf8]{inputenc}
	\usepackage[T1]{fontenc}
	\usepackage{graphics}                 % Packages to allow inclusion of graphics
	\usepackage{color}                    % For creating coloured text and background
\fi

\usepackage{enumitem}
\setlist{nolistsep}

\usepackage{amsmath,amssymb,amsfonts} % Typical maths resource packages
\usepackage[retainorgcmds]{IEEEtrantools}
\usepackage{caption}

\usepackage{tikz}
\usetikzlibrary{calc,arrows,decorations.pathmorphing,intersections}

\usepackage[font={small,sf},labelfont={bf},labelsep=endash]{caption}
\usepackage{sansmath}

\def\width{18}
\def\hauteur{11}

\oddsidemargin 0cm
\evensidemargin 0cm

\pagestyle{myheadings}         % Option to put page headers
                               % Needed \documentclass[a4paper,twoside]{article}
\markboth{{MEFT}}
{{\small\it \protect\input{../../LIFE.txt}}}

\addtolength{\hoffset}{-0.5cm}
\addtolength{\textwidth}{2.5cm}
\addtolength{\topmargin}{-1.5cm}
\addtolength{\textheight}{3cm}

%\textwidth 15.5cm
%\topmargin -1.5cm
\setlength{\parindent}{0pt}
\setlength{\parskip}{1ex  plus  0.5ex  minus  0.2ex}
%\parindent 0.5cm
%\textheight 25cm
%\parskip 1mm


% Math macros
\newcommand{\ud}{\,\mathrm{d}} 
\newcommand{\HRule}{\rule{\linewidth}{0.5mm}}

\author{Prof. Bernardo B. Carvalho} 

%%%%, Bernardo Brotas Carvalho\\bernardo.carvalho@tecnico.ulisboa.pt} 
\date{ Setembro 2015} 

\begin{document} 

%	\includegraphics[width=0.2\textwidth]{../logo-ist}%\\[1cm]  %%  Logo_IST_color

%	\HRule \\[0.5cm]
%	{ \huge \sf  \textsc{Construções Geométricas em Lentes Delgadas (aproximação paraxial)} }\\[0.4cm] % \bfseries 
%	{ \large \bfseries Determinação da constante de Planck.}\\

%	{ \large \bfseries Procedimento Experimental}\\
%	\HRule \\%[0.5cm]
%TURNO:________ GRUPO:________ DATA: ___ \\
%Número: _______ Nome: ________________ \\
%Número: _______ Nome: ________________ \\
{  \sf  Relatório da Experiência de Instrumentos Ópticos} %[0.4cm] % \bfseries 

\input{../../Nomes.txt}


\section{\sf Trabalho preparatório a realizar  ANTES da sessão de Laboratório:}

\subsection{\sf  Descreva por palavras suas quais os objectivos do trabalho que irá realizar na sessão de laboratório.}


\noindent\underline{\makebox[\textwidth][r]{~}} \\
\noindent\underline{\makebox[\textwidth][r]{~}} \\
\noindent\underline{\makebox[\textwidth][r]{~}} \\
\noindent\underline{\makebox[\textwidth][r]{~}} \\
\noindent\underline{\makebox[\textwidth][r]{~}} \\
\noindent\underline{\makebox[\textwidth][r]{~}} \\
\noindent\underline{\makebox[\textwidth][r]{~}} \\

% (uma folha A4). 
%Indique as expressões que irá utilizar para obter as grandezas experimentais, bem como as expressões para calcular as incertezas. Inclua esta parte também no Relatório. Este irá constituir o ÚNICO meio de consulta na Prova Individual.



\subsection{\sf Equações }
Escreva no seguinte quadro todas as equações necessárias para calcular as grandezas bem como as suas incertezas.
\begin{center}
\framebox[18cm]{\rule{0pt}{8.5cm}}
\end{center}


\section{\sf Relatório}
\subsection{\sf Montagem Experimental}
Desenhe um diagrama das diversas montagens experimentais que realizou. Inclua em anexo os esquemas de traçado de raios em papel milimétrico.

\begin{center}
\framebox[18cm]{\rule{0pt}{10cm}}
\end{center}

%%%%%%%%%%%%%%%%%%%%%%%%%%%%%%%%%%%%%%%%%%%%%%%%%

\subsection{\sf Microscópio composto}
Preencha as seguintes tabelas indicando  apenas os algarismos significativos. Terá que verificar as contas com auxílio da calculadora, para um dos ensaios e na presença do docente.
\vspace{1 cm}

Objectiva $f_{obj}=$~\underline{\makebox[1cm][r]{~}} mm;
Ocular $f_{ocu}=$~\underline{\makebox[1cm][r]{~}} mm; 
Ampliação angular $M_A=$~\underline{\makebox[1cm][r]{~}};\\
Objecto $h=$~\underline{\makebox[1cm][r]{~}} mm; 

\begin{center}
	%\centering
	\begin{tabular}{|c|c|c|c|c|c||c|c|}
	\hline
	 Ensaio   &  
	 $h'$  (mm) &  
	 $\epsilon_{h'}$ (mm) & 
	 $M_T$  & 
	 $h''$ (mm) & 
	 $\epsilon_{h''}$ (mm) & 
	 $h''/h$ & 
	 $M_T\times m_A$\\
	 
	\hline \hline
	1  &  &  & \quad\quad\quad\quad &  &  &  \makebox[1cm][r]{} &  \makebox[1cm][r]{} \\\cline{1-8}
	 2 &  &  & &  &  &  & \\ \cline{1-8}
	 3 &  &  &  &   & & &\\ \hline
			\end{tabular}
\end{center}


%%%%%%%%%%%%%%%%%%%%%%%%%%%%%%%%%%%%%%%%%%%%%%%%%

\newpage
\subsection{\sf Telescópio}


Objectiva $f_{obj}=$~\underline{\makebox[1cm][r]{~}} mm;
Ocular $f_{ocu}=$~\underline{\makebox[1cm][r]{~}} mm; 
$M=M_a=-f_{obj}/f_{ocu}=$~\underline{\makebox[1cm][r]{~}} 

\begin{center}
	\begin{tabular}{|c|c|c|c|c|}
	\hline
	 Pos. ocu. (mm)   &  
	 $\epsilon$  (mm) & 
	 Pos. obj. (mm) &  
	 $\epsilon$ (mm) &
	 Tam. img. / tam. obj. \\
	 
	\hline \hline
	 &  &  & & \makebox[1cm][r]{} \\ \hline
	\end{tabular}
\end{center}

%%%%%%%%%%%%%%%%%%%%%%%%%%%%%%%%%%%%%%%%%%%%%%%%%

\vspace{1 cm}

\subsection{\sf Goniómetro}

\subsubsection{\sf Ângulo do prisma}

\begin{center}
	\begin{tabular}{|c|c|c|c||c|}
	\hline
	Ensaio &
	 Esquerda    &  
	 Direita & 
	 Âng. $\alpha\pm\epsilon_{\alpha} (^{\circ})$ \\
	 
	\hline \hline
	 1 & $\quad^{\circ}\quad'\quad''$ & $\quad^{\circ}\quad'\quad''$ & $\makebox[2cm][r]{~}\pm\makebox[2cm][r]{~}$ \\ \cline{1-4}
	 2 & $\quad^{\circ}\quad'\quad''$ & $\quad^{\circ}\quad'\quad''$ & $\quad\pm\quad$ \\ \cline{1-4}
	 3 & $\quad^{\circ}\quad'\quad''$ & $\quad^{\circ}\quad'\quad''$ & $\quad\pm\quad$\\ \hline
	\end{tabular}
\end{center}

%%%%%%%%%%%%%%%%%%%%%%%%%%%%%%%%%%%%%%%
\vspace{0.5 cm}
\subsubsection{\sf Rede de difracção}

Rede: ~\underline{\makebox[1cm][r]{~}} linhas/mm;
Lâmpada espetral: ~\underline{\makebox[3cm][r]{~}}

\vspace{1 cm}
\underline{1.ª ordem}

\begin{center}
	\begin{tabular}{|c|c|c|c|c}
	\hline
	 Cor &
	 C.d.o. (nm) &
	 Esquerda    &  
	 Direita  \\
	\hline \hline
	  \makebox[3cm][r]{} & & $\quad^{\circ}\quad'\quad''$ & $\quad^{\circ}\quad'\quad''$ \\ \cline{1-3}
	 & & $\quad^{\circ}\quad'\quad''$ & $\quad^{\circ}\quad'\quad''$ \\ \cline{1-3}
	 & & $\quad^{\circ}\quad'\quad''$ & $\quad^{\circ}\quad'\quad''$ \\ \cline{1-3}
	 & & $\quad^{\circ}\quad'\quad''$ & $\quad^{\circ}\quad'\quad''$ \\ \cline{1-3}
	 & & $\quad^{\circ}\quad'\quad''$ & $\quad^{\circ}\quad'\quad''$ \\ \hline
	\end{tabular}
\end{center}

Poder de resolução
$\lambda_1$: ~\underline{\makebox[1cm][r]{~}} nm; $\lambda_2$: ~\underline{\makebox[1cm][r]{~}} nm; $R$: ~\underline{\makebox[1.5cm][r]{~}}

\vspace{1 cm}
\underline{2.ª ordem}

\begin{center}
	\begin{tabular}{|c|c|c|c|c}
	\hline
	 Cor &
	 C.d.o. (nm) &
	 Esquerda    &  
	 Direita  \\
	\hline \hline
	  \makebox[3cm][r]{} & & $\quad^{\circ}\quad'\quad''$ & $\quad^{\circ}\quad'\quad''$ \\ \cline{1-3}
	 & & $\quad^{\circ}\quad'\quad''$ & $\quad^{\circ}\quad'\quad''$ \\ \cline{1-3}
	 & & $\quad^{\circ}\quad'\quad''$ & $\quad^{\circ}\quad'\quad''$ \\ \cline{1-3}
	 & & $\quad^{\circ}\quad'\quad''$ & $\quad^{\circ}\quad'\quad''$ \\ \cline{1-3}
	 & & $\quad^{\circ}\quad'\quad''$ & $\quad^{\circ}\quad'\quad''$ \\ \hline
	\end{tabular}
\end{center}

%%%%%%%%%%%%%%%%%%%%%%%%%%%%%%%%%%%%%%%
\section{\sf Análise, Conclusões e Comentários}
\noindent\underline{\makebox[\textwidth][r]{~}} \\
\noindent\underline{\makebox[\textwidth][r]{~}} \\
\noindent\underline{\makebox[\textwidth][r]{~}} \\
\noindent\underline{\makebox[\textwidth][r]{~}} \\
\noindent\underline{\makebox[\textwidth][r]{~}} \\
\noindent\underline{\makebox[\textwidth][r]{~}} \\
\noindent\underline{\makebox[\textwidth][r]{~}} \\
\noindent\underline{\makebox[\textwidth][r]{~}} \\
\noindent\underline{\makebox[\textwidth][r]{~}} \\
\noindent\underline{\makebox[\textwidth][r]{~}} \\
\noindent\underline{\makebox[\textwidth][r]{~}} \\
\noindent\underline{\makebox[\textwidth][r]{~}} \\
\noindent\underline{\makebox[\textwidth][r]{~}} \\
\noindent\underline{\makebox[\textwidth][r]{~}} \\
\noindent\underline{\makebox[\textwidth][r]{~}} \\
\noindent\underline{\makebox[\textwidth][r]{~}} \\
\noindent\underline{\makebox[\textwidth][r]{~}} \\
\noindent\underline{\makebox[\textwidth][r]{~}} \\
\noindent\underline{\makebox[\textwidth][r]{~}} \\
\noindent\underline{\makebox[\textwidth][r]{~}} \\
\noindent\underline{\makebox[\textwidth][r]{~}} \\
\noindent\underline{\makebox[\textwidth][r]{~}} \\



%\newpage

\end{document} 