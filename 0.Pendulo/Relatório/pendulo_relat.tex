%&program=xelatex
%&encoding=UTF-8 Unicode
% SVN keywords
% $Author: bernardo $
% $Date: 2014-10-10 08:48:19 +0100 (Fri, 10 Oct 2014) $
% $Revision: 6700 $
% $URL: http://metis.ipfn.ist.utl.pt:8888/svn/cdaq/Users/Bernardo/Aulas/LFEB/teXfiles/Pendulo/pendulo_guia.tex $
\documentclass[a4paper,12pt]{article}      % Comments after  % are ignored
%\usepackage{hyperref}                 % For creating hyperlinks in cross references
%\documentclass[a4paper,12pt]{article} 
%
\usepackage{ifxetex}% for LATEX, or
\usepackage{ifplatform} 
\usepackage{verbatim}

\ifxetex
\usepackage{polyglossia} \setmainlanguage{portuges}
\usepackage{fontspec}
%
\ifwindows
\setmainfont{Garamond}
\setsansfont{Gill Sans MT}
\setmonofont[Scale=0.95]{Courier}
\fi
\iflinux
\setmainfont[Ligatures=TeX]{Linux Libertine O}
\setsansfont[Ligatures=TeX,Scale=MatchLowercase]{Linux Biolinum}
\setmonofont[Scale=0.95]{Courier}
\fi
\ifmacosx
% add settings
\fi
%
\usepackage{xcolor,graphicx} 
\else
%PdfLaTEX
\usepackage[portuguese]{babel}
%\usepackage[latin1]{inputenc}
\usepackage[utf8]{inputenc}
\usepackage[T1]{fontenc}
\usepackage{graphics}                 % Packages to allow inclusion of graphics
\usepackage{color}                    % For creating coloured text and background
\fi

\usepackage{amsmath,amssymb,amsfonts} % Typical maths resource packages

\oddsidemargin 0cm
\evensidemargin 0cm

\pagestyle{myheadings}         % Option to put page headers
                               % Needed \documentclass[a4paper,twoside]{article}
\markboth{{MEFT}}
{{\small\it \protect\input{../../LIFE.txt}}}

\textwidth 15.5cm
\topmargin -1cm
\parindent 0cm
\textheight 24cm
\parskip 1mm



% Math macros
\newcommand{\ud}{\,\mathrm{d}} 
\newcommand{\HRule}{\rule{\linewidth}{0.5mm}}

%\title{ e Difracção de Ondas Electromagnéticas num meio dieléctrico, homogéneo e isotrópico } 
%\subtitle{ aplicação à luz visível} 

\author{Prof. Bernardo B. Carvalho} 

%, Bernardo Brotas Carvalho\\bernardo@ipfn.ist.utl.pt} 
\date{ Setembro 2012} 

\begin{document} 


%	\begin{center}
%	\textsc{\large Laboratório de Física Experimental Básica - MEFT - 2012/2013 }\\%[0.5cm]
%	\end{center}
\includegraphics[width=0.2\textwidth]{../../logo-ist}%\\[1cm]  %%  Logo_IST_color
	
	\HRule \\[0.5cm]
	{ \huge   \bfseries \textsc{ Pendulo Gravítico } }\\[0.4cm]
	{ \large \bfseries Determinação do período do pêndulo simples e aferição com o valor da Aceleração da Gravidade local $g$  }\\
%	{ \large \begin{flushleft}
%	 $\bullet$ Deflexão magnética \\
%	 $\bullet$ Deflexão magnética em equilíbrio com deflexão eléctrica
%	\end{flushleft} }
	\HRule \\%[0.5cm]
%	\textsc{\Large Laboratório de Física Experimental Básica}\\[0.5cm]
	
%	\input{./title_exa.tex} 

%\maketitle
%\section{\sf  Conceitos necessários:} 
%\begin{enumerate}
%	\item Força eléctrica. Campo eléctrico (Electrostático)
%	\item Potencial eléctrico. Equipotencial. Energia potencial eléctrica 
%	\item Condutores e dieléctricos. Condensador plano
%	\item Efeitos da corrente eléctrica estacionária criada por uma espira 	
%	\item Força de Laplace
%\end{enumerate}


\section{\sf Procedimento Experimental}
{ \large Material }
 \begin{flushleft}
	 $\bullet$ Suporte do Pêndulo \\
	 $\bullet$ Massas de Chumbo, linha inextensível e com massa desprezável \\
	 $\bullet$ Régua graduada, Cronómetro, Fita métrica, transferidor, balança
\end{flushleft} 

Comece a sessão de laboratório por estimar a precisão que obtém na medição do tempo com o cronómetro, tendo em conta 
o tempo de reacção do corpo humano. 
Com a ajuda de um colega e de uma régua graduada obtenha 15 medidas da queda da régua e a partir da média e desvio padrão obtenha o seu tempo de reação e a incerteza \footnote{$\overline{t}=\sqrt{\frac{2 \overline{D}}{g}}$ e   
$e_{\overline{t}}=\sqrt{\frac{2 }{g}} \cdot \frac{1}{2\sqrt{\overline{D}}} \cdot e_{\overline{D}}  
= \overline{t} \cdot \frac{1}{2\overline{D}} \cdot e_{\overline{D}} $ }

\begin{tabular}{|r|c|c|c|}
\hline
Ensaio  & A - Distância & B - Distância & C - Distância  \\
\# & de queda (cm) & de queda (cm) & de queda (cm)\\
\hline \hline
1 & & & \\
\hline
2 & &  &\\
\hline 3 & & & \\
\hline 4 & & & \\
\hline 5 & & & \\
\hline \hline
Média $\overline{D}$ (m) & &  & \\
Desvio padrão (m) & & & \\
Erro da Média  $e_{\overline{D}}$ (m) & & & \\ 
Tempo de reação $\overline{t} \pm e_{\overline{t}}$ (s) & $\qquad \pm$ \quad(m) & $\qquad \pm$ \quad (m) & $\qquad \pm$ \quad (m) \\
\hline
\end{tabular}

Monte o sistema de pêndulo gravítico e obtenha o seu período para diversos comprimentos do fio. 
Obtenha o valor de $g_{exp}$ para estes ensaios, usando a expressão (9) do texto de apoio, bem como a respectiva incerteza experimental. 
Compare o valor final de $g_{exp}$ obtido com o valor tabelado $g_{tab}$ e estime o desvio à exactidão que obteve. 

\smallskip

Tenha em atenção os seguintes aspectos:
 \begin{flushleft}
	 $\bullet$ Uma massa pendurada num fio tem mais que o grau de liberdade em $\theta$. Tente assegurar-se que o pêndulo oscila apenas ao longo de um plano. \\
	 $\bullet$ Tente minimizar o efeitos de paralaxe na determinação do ângulo máximo, quer no posição usada para cronometrar um período  \\
	 $\bullet$ Naturalmente a massa utilizada, não é pontual. Qual é o efeito  na medida e incerteza do comprimento $l$? \\	
	 $\bullet$ Como pode minimizar a incerteza da medição do período? Usando um ou $N$ ciclos? \\
\end{flushleft} 


Actividades adicionais, se tiver tempo:
 \begin{flushleft}
	 $\bullet$ Utilize a montagem para medição precisa do Periodo. Compare com os outros resultados.\\
	 $\bullet$ Verifique experimentalmente que o período do pêndulo não depende do valor da massa.\\
	 $\bullet$ Verifique experimentalmente a dependência do ângulo máximo no período do pêndulo. 
	 Para que valores de $\theta_0$ o valor calculado de $g_{exp}$ se afasta de $\overline{g_{exp}}$ com desvio de $0.05$ ?\\
	 $\bullet$ Tente estimar a percentagem de energia devido ao atrito que se perde em cada ciclo .
\end{flushleft} 

 


%\textbf{Ver descrição no guia de laboratório pgs.77-81}

\end{document} 	

