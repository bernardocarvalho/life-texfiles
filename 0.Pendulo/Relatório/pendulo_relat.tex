%&program=xelatex
%&encoding=UTF-8 Unicode
% SVN keywords
% $Author: bernardo $
% $Date: 2014-10-10 08:48:19 +0100 (Fri, 10 Oct 2014) $
% $Revision: 6700 $
% $URL: http://metis.ipfn.ist.utl.pt:8888/svn/cdaq/Users/Bernardo/Aulas/LFEB/teXfiles/Pendulo/pendulo_guia.tex $
\documentclass[a4paper,12pt]{article}      % Comments after  % are ignored
%\usepackage{hyperref}                 % For creating hyperlinks in cross references
%\documentclass[a4paper,12pt]{article} 
%
\usepackage{ifxetex}% for LATEX, or
\usepackage{ifplatform} 
\usepackage{verbatim}

\ifxetex
\usepackage{polyglossia} \setmainlanguage{portuges}
\usepackage{fontspec}
%
\ifwindows
\setmainfont{Garamond}
\setsansfont{Gill Sans MT}
\setmonofont[Scale=0.95]{Courier}
\fi
\iflinux
\setmainfont[Ligatures=TeX]{Linux Libertine O}
\setsansfont[Ligatures=TeX,Scale=MatchLowercase]{Linux Biolinum}
\setmonofont[Scale=0.95]{Courier}
\fi
\ifmacosx
% add settings
\fi
%
\usepackage{xcolor,graphicx} 
\else
%PdfLaTEX
\usepackage[portuguese]{babel}
%\usepackage[latin1]{inputenc}
\usepackage[utf8]{inputenc}
\usepackage[T1]{fontenc}
\usepackage{graphics}                 % Packages to allow inclusion of graphics
\usepackage{color}                    % For creating coloured text and background
\fi

\usepackage{amsmath,amssymb,amsfonts} % Typical maths resource packages

\oddsidemargin 0cm
\evensidemargin 0cm

\pagestyle{myheadings}         % Option to put page headers
                               % Needed \documentclass[a4paper,twoside]{article}
\markboth{{MEFT}}
{{\small\it \protect\input{../../LIFE.txt}}}

\textwidth 15.5cm
\topmargin -1cm
\parindent 0cm
\textheight 24cm
\parskip 1mm



% Math macros
\newcommand{\ud}{\,\mathrm{d}} 
\newcommand{\HRule}{\rule{\linewidth}{0.5mm}}

%\title{ e Difracção de Ondas Electromagnéticas num meio dieléctrico, homogéneo e isotrópico } 
%\subtitle{ aplicação à luz visível} 

\author{Prof. Bernardo B. Carvalho} 

%, Bernardo Brotas Carvalho\\bernardo@ipfn.ist.utl.pt} 
\date{ Setembro 2017} 

\begin{document} 


%	\begin{center}
%	\textsc{\large Laboratório de Física Experimental Básica - MEFT - 2012/2013 }\\%[0.5cm]
%	\end{center}
\includegraphics[width=0.2\textwidth]{../../logo-ist}%\\[1cm]  %%  Logo_IST_color
	
	\HRule \\[0.5cm]
	{ \huge   \bfseries \textsc{ Pendulo Gravítico } }\\[0.4cm]
	{ \large \bfseries Determinação do período do pêndulo simples e aferição com o valor da Aceleração da Gravidade local $g$  }\\
%	{ \large \begin{flushleft}
%	 $\bullet$ Deflexão magnética \\
%	 $\bullet$ Deflexão magnética em equilíbrio com deflexão eléctrica
%	\end{flushleft} }
	\HRule \\%[0.5cm]
%	\textsc{\Large Laboratório de Física Experimental Básica}\\[0.5cm]
	

\section*{\sf Procedimento Experimental}
{ \large Material }
 \begin{flushleft}
	 $\bullet$ Suporte do Pêndulo. \\
	 $\bullet$ Massas de chumbo, linha inextensível e com massa desprezável. \\
	 $\bullet$ Régua graduada, cronómetro, fita métrica, transferidor, balança.
\end{flushleft} 

Comece a sessão de laboratório por estimar o atraso e a precisão que obtém na medição do tempo com o cronómetro, tendo em conta 
o tempo de reacção do sistema nervoso. 
Para cada membro do grupo, com  uma régua graduada e a ajuda de um(a) colega obtenha 15 medidas da queda da régua. A partir da média e desvio padrão obtenha o  tempo de reação  e a incerteza. 

$\overline{t}=\sqrt{\frac{2 \overline{D}}{g}}$ e   
$\sigma_{\overline{t}}=\sqrt{\frac{2 }{g}} \cdot \frac{1}{2\sqrt{\overline{D}}} \cdot \sigma_{\overline{D}}  
= \overline{t} \cdot \frac{1}{2\overline{D}} \cdot \sigma_{\overline{D}} $ 

%\footnote{\, 


\begin{center}
\begin{footnotesize}
\begin{tabular}{|r|c|c|c|}
\hline
Ensaio  & A - Distância & B - Distância & C - Distância  \\
\# & de queda [cm] & de queda [cm] & de queda [cm]\\
\hline \hline
1 & & & \\
\hline
2 & &  &\\
\hline 3 & & & \\
\hline 4 & & & \\
\hline 5 & & & \\
\hline 6 & & & \\
\hline 7 & & & \\
\hline 8 & & & \\
\hline 9 & & & \\
\hline 10 & & & \\
\hline 11 & & & \\
\hline 12 & & & \\
\hline 13 & & & \\
\hline 14 & & & \\
\hline 15 & & & \\
\hline \hline
Média $\overline{D}$ [m] & &  & \\ \hline
Desvio padrão  $\sigma_{\overline{D}}$ [m] & & & \\ \hline
%Erro da Média  $e_{\overline{D}}$ [m] & & & \\ 
%Tempo de reação $\overline{t} \pm e_{\overline{t}}$ [s] & $\qquad \pm$ \quad (s) & $\qquad \pm$ \quad (s) & $\qquad \pm$ \quad (s) \\
Tempo de reação $\overline{t}$ [s] & & & \\ \hline % \pm e_{\overline{t}}$ 
Desvio padrão  $\sigma_{\overline{t}}$ [s] & & & \\
\hline

\end{tabular}
\end{footnotesize}
\end{center}


Monte o sistema de pêndulo gravítico e obtenha o seu período para diversos comprimentos do fio $L$, usando a medição de $N$ ciclos. 
A medição é de um intervalo de tempo e como tal os atrasos da reação compensam no início e no fim da contagem. No entando deve considerar como  
erro de medição o dobro do desvio padrão. Para o erro da média $\overline{\Delta t}$ deve considerar o majorante entre este erro, $2 \sigma_{\overline{t}}$ e o maior desvio entre o valor  $\overline{\Delta  t} $ e cada ensaio individual. 

Obtenha o valor de $g_{exp}$ para estes ensaios, usando a expressão (9) do texto de apoio, bem como a respectiva incerteza experimental. 
Compare o valor final de $g_{exp}$ obtido com o valor tabelado $g_{tab}$ para Lisboa e estime o desvio à exactidão que obteve. 

\begin{center}
\begin{small}
	\noindent Angulo inicial:	$\theta \simeq$ \underline{\makebox[1.5cm][r]{~}} rad,  Número de ciclos: $\quad N=$\underline{\makebox[1cm][r]{~}} \\
\bigskip
\begin{tabular}{|r|c|c|c|c|}
\hline
Ensaio  \# & L: $\qquad \pm$ \qquad [m] & L: $\qquad \pm$ \qquad[m] & L: $\qquad \pm$ \qquad[m] & L: $\qquad \pm$ \qquad[m]\\
%\# & de queda [cm] & de queda [cm] & de queda [cm]\\
\hline \hline
$\Delta t$ A  [s]&  $\qquad \pm$ \quad &  $\qquad \pm$ \quad&  $\qquad \pm$ \quad&  $\qquad \pm$ \quad\\ \hline
$\Delta t$ B  [s]&  $\qquad \pm$ \quad &  $\qquad \pm$ \quad&  $\qquad \pm$ \quad&  $\qquad \pm$ \quad\\ \hline
$\Delta t$ C  [s]&  $\qquad \pm$ \quad &  $\qquad \pm$ \quad&  $\qquad \pm$ \quad&  $\qquad \pm$ \quad\\ \hline
\hline 
Média $\overline{\Delta t}$ [s] & $\qquad \pm$ \quad  & $\qquad \pm$ \quad  &  $\qquad \pm$ \quad  & $\qquad \pm$ \quad \\
Período $\overline{T}$ [s] & $\qquad \pm$ \quad  & $\qquad \pm$ \quad  &  $\qquad \pm$ \quad  &  $\qquad \pm$ \quad  \\
 $\overline{g}$ [ms$^{-2}$] & $\qquad \pm$ \quad  & $\qquad \pm$ \quad  & $\qquad \pm$ \quad & $\qquad \pm$ \quad \\
%Erro da Média  $e_{\overline{D}}$ [m] & & & \\ 
% $\overline{t} \pm e_{\overline{t}}$ [s] & $\qquad \pm$ \quad [s] & $\qquad \pm$ \quad [s] & $\qquad \pm$ \quad [s] \\
\hline

\end{tabular}
\end{small}
\end{center}

Tenha em atenção os seguintes aspectos e comente-os na discussão final:
 \begin{flushleft}
	 $\bullet$ Utilize apenas algarismos significativos (a.s.) nas tabelas. O erros devem conter no máximo 2 a.s. \\
	 $\bullet$ Qual a vantagem de usar na medição $N$ ciclos do pêndulo? \\
	 $\bullet$ Naturalmente a massa utilizada não é pontual. Qual é o efeito na medida e incerteza do comprimento $L$? \\	
	 $\bullet$ Uma massa pendurada num fio tem mais que o grau de liberdade em $\theta$. Tente assegurar que o pêndulo oscila apenas ao longo de um plano vertical. \\
	 $\bullet$ Tente minimizar o efeitos de paralaxe na determinação do ângulo máximo.  \\
	 $\bullet$ Qual a posição do pêndulo que usa para cronometrar o intervalo de tempo?  \\
\end{flushleft} 

\subsubsection*{\sf Resultados}
%\bigskip
%	\noindent
$g_{exp}=$ \underline{\makebox[1.5cm][r]{~}}  $\pm$  	\underline{\makebox[1cm][r]{~}} [ms$^{-2}$]

\noindent  
Desvio à Exatidão $=$~\underline{\makebox[1cm][r]{~}}\%, 
Incerteza relativa $=$~\underline{\makebox[1cm][r]{~}}\% 


%\smallskip



\subsubsection*{\sf Actividades adicionais, se tiver tempo}
 \begin{flushleft}
	 $\bullet$ Utilize a montagem electrónica com barreira óptica para medição precisa do período. Compare com os outros resultados.\\
	 $\bullet$ Verifique experimentalmente que o período do pêndulo não depende do valor da massa.\\
	 $\bullet$ Verifique experimentalmente a alteração do período do pêndulo para ângulos iniciais grandes. 
	 Para que valores de $\theta_0$ o valor calculado de $g$’ se afasta de $g_{exp}$ com desvio $\ge 5 \%$ ?\\
	 $\bullet$ Tente estimar a percentagem de energia devido ao atrito que se perde em cada ciclo .
\end{flushleft} 

\newpage

 \subsection*{\sf Análise, conclusões e comentários finais}
\noindent\underline{\makebox[\textwidth][r]{~}} \\
\noindent\underline{\makebox[\textwidth][r]{~}} \\
\noindent\underline{\makebox[\textwidth][r]{~}} \\
\noindent\underline{\makebox[\textwidth][r]{~}} \\
\noindent\underline{\makebox[\textwidth][r]{~}} \\
\noindent\underline{\makebox[\textwidth][r]{~}} \\
\noindent\underline{\makebox[\textwidth][r]{~}} \\
\noindent\underline{\makebox[\textwidth][r]{~}} \\
\noindent\underline{\makebox[\textwidth][r]{~}} \\
\noindent\underline{\makebox[\textwidth][r]{~}} \\
\noindent\underline{\makebox[\textwidth][r]{~}} \\
\noindent\underline{\makebox[\textwidth][r]{~}} \\
\noindent\underline{\makebox[\textwidth][r]{~}} \\
\noindent\underline{\makebox[\textwidth][r]{~}} \\
\noindent\underline{\makebox[\textwidth][r]{~}} \\
\noindent\underline{\makebox[\textwidth][r]{~}} \\
\noindent\underline{\makebox[\textwidth][r]{~}} \\
\noindent\underline{\makebox[\textwidth][r]{~}} \\
\noindent\underline{\makebox[\textwidth][r]{~}} \\
%\noindent\underline{\makebox[16.2cm][r]{~}} \\



\end{document} 	

