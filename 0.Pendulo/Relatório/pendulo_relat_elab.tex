%&program=xelatex
%&encoding=UTF-8 Unicode
% SVN keywords
% $Author: bernardo $
%
\documentclass[a4paper,12pt]{article}      % Comments after  % are ignored
%\usepackage{hyperref}                 % For creating hyperlinks in cross references
%\documentclass[a4paper,12pt]{article} 
%
\usepackage{ifxetex}% for LATEX, or
% Compiled with  XeTeX, Version 3.14159265-2.6-0.99999 (MiKTeX 2.9.6800 64-bit)

\usepackage{ifplatform} 
\usepackage{verbatim}

\ifxetex
\usepackage{polyglossia} \setmainlanguage{portuges}
\usepackage{fontspec}
%
\ifwindows
\setmainfont{Garamond}
%\setsansfont{Gill Sans MT}
\setmonofont[Scale=0.95]{Courier}
\fi
\iflinux
\setmainfont[Ligatures=TeX]{Linux Libertine O}
\setsansfont[Ligatures=TeX,Scale=MatchLowercase]{Linux Biolinum}
\setmonofont[Scale=0.95]{Courier}
\fi
\ifmacosx
% add settings
\fi
%
\usepackage{xcolor,graphicx} 
\else
%PdfLaTEX
\usepackage[portuguese]{babel}
%\usepackage[latin1]{inputenc}
\usepackage[utf8]{inputenc}
\usepackage[T1]{fontenc}
\usepackage{graphics}                 % Packages to allow inclusion of graphics
\usepackage{color}                    % For creating coloured text and background
\fi

\usepackage{amsmath,amssymb,amsfonts} % Typical maths resource packages

\oddsidemargin 0cm
\evensidemargin 0cm

\pagestyle{myheadings}         % Option to put page headers
                               % Needed \documentclass[a4paper,twoside]{article}
\markboth{{MEFT}}
{{\small\it \protect\input{../../LIFE.txt}}}

\textwidth 15.5cm
\topmargin -1cm
\parindent 0cm
\textheight 24cm
\parskip 1mm



% Math macros
\newcommand{\ud}{\,\mathrm{d}} 
\newcommand{\HRule}{\rule{\linewidth}{0.5mm}}

%\title{ e Difracção de Ondas Electromagnéticas num meio dieléctrico, homogéneo e isotrópico } 
%\subtitle{ aplicação à luz visível} 

\author{Prof. Bernardo B. Carvalho} 

%, Bernardo Brotas Carvalho\\bernardo@ipfn.ist.utl.pt} 
\date{ Setembro 2019} 

\begin{document} 
{  \sf  Relatório da Experiência do Pêndulo Gravítico} %[0.4cm] % \bfseries 



%\input{../../Nomes.txt}


\section{\sf Trabalho preparatório a realizar ANTES da sessão de Laboratório:}


\subsection{\sf Objectivos do Trabalho}
%\begin{enumerate}
 \begin{flushleft}
	 $\bullet$  Descreva sucintamente quais os objectivos do trabalho que irá realizar e que tipo de equipamentos irá utilizar.
 \end{flushleft}
%\end{enumerate}
\noindent\underline{\makebox[\textwidth][r]{~}} \\
\noindent\underline{\makebox[\textwidth][r]{~}} \\
\noindent\underline{\makebox[\textwidth][r]{~}} \\
\noindent\underline{\makebox[\textwidth][r]{~}} \\
\noindent\underline{\makebox[\textwidth][r]{~}} \\

%\subsubsection{\sf Equações }
%Escreva no seguinte quadro todas as equações necessárias para calcular as grandezas, bem como as suas incertezas.
%%\begin{center}
%%\framebox[13cm]{\rule{0pt}{6.5cm}}
%%\end{center}
%
%\fbox{\begin{minipage}{36em}
%$\overline{t}=\sqrt{\frac{2 \overline{D}}{g}}$\\
%$\sigma_{\overline{t}}=\sqrt{\frac{2 }{g}} \cdot \frac{1}{2\sqrt{\overline{D}}} \cdot \sigma_{\overline{D}}  
%= \overline{t} \cdot \frac{1}{2\overline{D}} \cdot \sigma_{\overline{D}} $ 
%\\
%\\
%\\
%\\
%\\
%\\
%\\
%\\
%\\
%\\
%\\
%\\
%
%\end{minipage}}
%
%\newpage


\section{\sf Procedimento Experimental}
{ \large Material }
 \begin{flushleft}
	 $\bullet$	Material utilizado e suas dimensões. Inclua uma fotografia do dispositivo empregue.  \\
%	 $\bullet$ Massas de chumbo, linha inextensível e com massa desprezável. \\
%	 $\bullet$ Régua graduada, cronómetro, fita métrica, transferidor, balança.
\end{flushleft} 

\subsubsection*{\sf Determinação do tempo de reação - incerteza da medida}
Obtenha uma aplicação na internet que lhe permita determinar O SEU tempo de reação. Efetue várias medidas e calcule o desvio padrão de modo a determinar a precisão com que conseguirá medir futuramente um intervalo de tempo com o cronómetro.

%Comece a sessão de laboratório por estimar o atraso e a precisão que obtém na medição do tempo com o cronómetro, tendo em conta o tempo de reacção do sistema nervoso. 
%Para cada membro do grupo, com  uma régua graduada e a ajuda de um(a) colega obtenha 15 medidas da queda da régua. A partir da média e desvio padrão obtenha o  tempo de reação  %e a incerteza. 


%\footnote{\, 


\begin{center}
\begin{footnotesize}
\begin{tabular}{|r|c|}
\hline
Ensaio  & Tempo de reação  \\
\# &  [ms] \\
\hline \hline
1 &  \\
\hline
2 & \\
\hline ... & \\
%\hline 4 & & & \\
%\hline 5 & & & \\
%\hline 6 & & & \\
%\hline 7 & & & \\
%\hline 8 & & & \\
%\hline 9 & & & \\
%\hline 10 & & & \\
\hline 11 &  \\
\hline \hline
%Erro da Média  $e_{\overline{D}}$ [m] & & & \\ 
%Tempo de reação $\overline{t} \pm e_{\overline{t}}$ [s] & $\qquad \pm$ \quad (s) & $\qquad \pm$ \quad (s) & $\qquad \pm$ \quad (s) \\
Tempo de reação $\overline{t}$ [ms] &  \\ \hline % \pm e_{\overline{t}}$ 
Desvio padrão  $\sigma_{\overline{t}}$ [ms] & \\
\hline

\end{tabular}
\end{footnotesize}
\end{center}

\subsubsection*{\sf Determinação do período do pêndulo}
Monte o sistema de pêndulo gravítico e obtenha o seu período $T$ para diversos comprimentos do fio $L$, usando a medição de $N$ ciclos. 
\emph{A medição é de um intervalo de tempo, $\Delta t$ e, como tal, os atrasos da reação compensam no início e no fim da contagem.} No entanto, deve considerar como como incerteza na medida do tempo total o dobro do desvio padrão obtido anteriormente na determinação do seu tempo de reação.
%. Para o erro da média $\overline{\Delta t}$ deve considerar o majorante entre este erro, $2 \sigma_{\overline{t}}$ e o maior desvio entre o valor  $\overline{\Delta  t} $ e cada ensaio individual. 

%Obtenha o valor de $g_{exp}$ para estes ensaios, usando a expressão (9) do texto de apoio, bem como a respectiva incerteza experimental. Compare o valor final de $g_{exp}$ obtido com o valor tabelado $g_{tab}$ para Lisboa e estime o desvio à exactidão que obteve. 

\begin{center}
\begin{small}
	\noindent Angulo inicial:	$\theta \simeq$ \underline{\makebox[1.5cm][r]{~}} rad,  Número de ciclos: $\quad N=$\underline{\makebox[1cm][r]{~}} \\
\bigskip
\begin{tabular}{|r|c|c|c|c|}
\hline
Ensaio  \# & L: $\qquad \pm$ \qquad [mm] & L: $\qquad \pm$ \qquad[mm] & L: $\qquad \pm$ \qquad[mm] & L: $\qquad \pm$ \qquad[mm]\\
\hline \hline
$\Delta t_1$   [s]&  $\qquad \pm$ \quad &  $\qquad \pm$ \quad&  $\qquad \pm$ \quad&  $\qquad \pm$ \quad\\ \hline
$\Delta t_2$   [s]&  $\qquad \pm$ \quad &  $\qquad \pm$ \quad&  $\qquad \pm$ \quad&  $\qquad \pm$ \quad\\ \hline
$\Delta t_2$   [s]&  $\qquad \pm$ \quad &  $\qquad \pm$ \quad&  $\qquad \pm$ \quad&  $\qquad \pm$ \quad\\ \hline
\hline 
Média $\overline{\Delta t}$ [s] & $\qquad \pm$ \quad  & $\qquad \pm$ \quad  &  $\qquad \pm$ \quad  & $\qquad \pm$ \quad \\
Período $\overline{T}$ [s] & $\qquad \pm$ \quad  & $\qquad \pm$ \quad  &  $\qquad \pm$ \quad  &  $\qquad \pm$ \quad  \\
% $\overline{g}$ [ms$^{-2}$] & $\qquad \pm$ \quad  & $\qquad \pm$ \quad  & $\qquad \pm$ \quad & $\qquad \pm$ \quad \\
%Erro da Média  $e_{\overline{D}}$ [m] & & & \\ 
% $\overline{t} \pm e_{\overline{t}}$ [s] & $\qquad \pm$ \quad [s] & $\qquad \pm$ \quad [s] & $\qquad \pm$ \quad [s] \\
\hline

\end{tabular}
\end{small}
\end{center}

\subsubsection*{\sf Determinação da relação entre o período do Pêndulo e o comprimento }
Estabeleça graficamente a relação entre o período e o comprimento do cabo, ajustando uma função adequada  de acordo com o modelo dimensional estabelecido. 

\subsubsection*{\sf Discussão }
 
Questione o seu trabalho de acordo com as instruções de revisão – \emph{peer review}.


Tenha em atenção os seguintes aspectos e comente-os:
 \begin{flushleft}
	 $\bullet$ Utilize apenas algarismos significativos (a.s.) nas tabelas. O erros devem conter no máximo 2 a.s. \\
	 $\bullet$ Qual a vantagem de usar na medição $N$ ciclos do pêndulo? \\
	 $\bullet$  Qual é a incerteza na medida do comprimento $L$? \\	
	 $\bullet$ Uma massa pendurada num fio tem mais que o grau de liberdade. Tente assegurar que o pêndulo oscila apenas no plano vertical. \\
	 $\bullet$ Tente minimizar o efeitos de paralaxe na determinação do ângulo máximo.  \\
	 $\bullet$ Qual a posição do pêndulo que usa para cronometrar o intervalo de tempo?  \\
\end{flushleft} 

%\subsubsection*{\sf Resultados finais}
%%\bigskip
%%	\noindent
%$g_{exp}=$ \underline{\makebox[1.5cm][r]{~}}  $\pm$  	\underline{\makebox[1cm][r]{~}} [ms$^{-2}$]
%
%\noindent  
%Desvio à exatidão $=$~\underline{\makebox[1cm][r]{~}}\%, 
%Incerteza relativa $=$~\underline{\makebox[1cm][r]{~}}\% 


%\smallskip



\subsubsection*{\sf Actividades adicionais, se tiver tempo}
 \begin{flushleft}
%	 $\bullet$ Utilize uma montagem electrónica com barreira óptica para medição precisa do período. Compare com os outros resultados.\\
	 $\bullet$ Verifique experimentalmente que o período do pêndulo não depende do valor da massa.\\
%	 $\bullet$ Verifique experimentalmente a alteração do período do pêndulo para ângulos iniciais grandes. 
%	 Para que valores de $\theta_0$ o valor calculado de $g$’ se afasta de $g_{exp}$ com desvio $\ge 5 \%$ ?\\
	 $\bullet$ Tente determinar para que valores da posição inicial o período se desvia mais do que $ 5 \%$  do valor médio para pequenas oscilações.\\
	 $\bullet$ Tente estimar a percentagem de energia devido ao atrito que se perde em cada ciclo.
\end{flushleft} 

\newpage

 \subsection*{\sf Análise, conclusões e comentários finais}
\noindent\underline{\makebox[\textwidth][r]{~}} \\
\noindent\underline{\makebox[\textwidth][r]{~}} \\
\noindent\underline{\makebox[\textwidth][r]{~}} \\
\noindent\underline{\makebox[\textwidth][r]{~}} \\
\noindent\underline{\makebox[\textwidth][r]{~}} \\
\noindent\underline{\makebox[\textwidth][r]{~}} \\
\noindent\underline{\makebox[\textwidth][r]{~}} \\
\noindent\underline{\makebox[\textwidth][r]{~}} \\
\noindent\underline{\makebox[\textwidth][r]{~}} \\
\noindent\underline{\makebox[\textwidth][r]{~}} \\
\noindent\underline{\makebox[\textwidth][r]{~}} \\
\noindent\underline{\makebox[\textwidth][r]{~}} \\
\noindent\underline{\makebox[\textwidth][r]{~}} \\
\noindent\underline{\makebox[\textwidth][r]{~}} \\
\noindent\underline{\makebox[\textwidth][r]{~}} \\
\noindent\underline{\makebox[\textwidth][r]{~}} \\
\noindent\underline{\makebox[\textwidth][r]{~}} \\
\noindent\underline{\makebox[\textwidth][r]{~}} \\
\noindent\underline{\makebox[\textwidth][r]{~}} \\
%\noindent\underline{\makebox[16.2cm][r]{~}} \\



\end{document} 	

