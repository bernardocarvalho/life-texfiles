%%&program=xelatex
%&encoding=UTF-8 Unicode
% SVN keywords
% $Author$
% $Date$
% $Revision$
% $URL$
\documentclass[a4paper,12pt]{article}  % Comments after  % are ignored
%\usepackage{hyperref}                 % For creating hyperlinks in cross references
%
\usepackage{ifxetex}% for XELATEX, or PDFlatex
\usepackage{ifplatform} 
%
\ifxetex
	\usepackage{polyglossia} \setmainlanguage{portuges}
	\usepackage{fontspec}
	\ifwindows
		\setmainfont[Ligatures=TeX]{Garamond}
		\setsansfont[Ligatures=TeX]{Gill Sans MT}
		\setmonofont{Consolas}
%		\setmonofont[Scale=MatchLowercase]{Courier}
	\fi
	\iflinux
		\setmainfont[Ligatures=TeX]{Linux Libertine O}
		\setsansfont[Ligatures=TeX,Scale=MatchLowercase]{Linux Biolinum}
		\setmonofont[Scale=MatchLowercase]{Courier}
	\fi
	\ifmacosx
	% add settings
	% Use xelatex -no-shell ...
	\fi
	\usepackage{xcolor,graphicx} 
\else
	\usepackage[portuguese]{babel}
	%\usepackage[latin1]{inputenc}
	\usepackage[utf8]{inputenc}
	\usepackage[T1]{fontenc}
	\usepackage{graphics}                 % Packages to allow inclusion of graphics
	\usepackage{color}                    % For creating coloured text and background
\fi

\usepackage{enumitem}
\setlist{nolistsep}

\usepackage{amsmath,amssymb,amsfonts} % Typical maths resource packages
\usepackage[retainorgcmds]{IEEEtrantools}
\usepackage{caption}


\oddsidemargin 0cm
\evensidemargin 0cm

\pagestyle{myheadings}         % Option to put page headers
                               % Needed \documentclass[a4paper,twoside]{article}
\markboth{{\small \it  Laboratório de Física Experimental Básica}}
{{\small\it MEFT - 1º Sem. 2014/2015} }

\addtolength{\hoffset}{-0.5cm}
\addtolength{\textwidth}{2.5cm}
\addtolength{\topmargin}{-1.5cm}
\addtolength{\textheight}{3cm}

%\textwidth 15.5cm
%\topmargin -1.5cm
\setlength{\parindent}{0pt}
\setlength{\parskip}{1ex  plus  0.5ex  minus  0.2ex}
%\parindent 0.5cm
%\textheight 25cm
%\parskip 1mm


% Math macros
\newcommand{\ud}{\,\mathrm{d}} 
\newcommand{\HRule}{\rule{\linewidth}{0.5mm}}

\author{Prof. Bernardo B. Carvalho} 

%%%%, Bernardo Brotas Carvalho\\bernardo.carvalho@tecnico.ulisboa.pt} 
\date{ Outubro 2014} 

\begin{document} 

	\includegraphics[width=0.2\textwidth]{../logo-ist}%\\[1cm]  %%  Logo_IST_color

	\HRule \\[0.5cm]
	{ \huge \sf  \textsc{Efeito fotoeléctrico}} \\[0.4cm] % \bfseries 
%	{ \huge \sf  \textsc{Construções Geométricas em Lentes Delgadas (aproximação paraxial)} }\\[0.4cm] % \bfseries 
	{ \large \bfseries Determinação da constante de Planck.}\\
%	{ \large \bfseries Procedimento Experimental}\\
	\HRule \\%[0.5cm]

\section{\sf Princípio do método}
Electrões podem ser emitidos da superfície de alguns metais, quando esta é iluminada com luz
de comprimento de onda suficientemente curto (efeito fotoeléctrico). A energia cinética dos
electrões emitidos (fotoelectrões) depende da frequência da luz incidente, mas não da sua
intensidade (a intensidade só determina o número de fotoelectrões emitidos). Esta verificação
experimental contraria os princípios da física clássica, e a sua explicação correcta só foi
proposta em 1905 por Einstein. Einstein postulou que a luz era constituída por um fluxo de
fotões, cada um com uma energia dependente da frequência,


Nesse caso  (Figura \ref{fig:angulo}) o índice de refração, $n$, pode ser calculado simplesmente através da expressão seguinte: 

\begin{equation}
	\label{eq:energia}
	E= h \mu
\end{equation}

~
\newpage
\subsection{\sf Questões a responder ANTES da sessão de Laboratório:}
\begin{enumerate}
\item Obtenha uma imagem típica da dispersão da luz Branca num prisma triangular. O índice de refração, $n(\lambda)$, é uma função crescente ou decrescente?
\item Nessa figura de dispersão como faria para  identificar qual é a côr que está na posição de \emph{desvio mínimo}?
\item Se na montagem de laboratório substituir a lampada de descarga por uma de incandescência que imagem obteria com o goniometro?. 
\item O espectro de emissão do Hidrogéneo, na série de Balmer (transição $3 \to 2$) tem duas riscas no vermelho, respetivamente a $\lambda = 656.272\, nm$ e $\lambda = 656.2852\,nm$.
Qual a Resolução mínima de um instrumento (Espectrómetro) capaz de distinguir estas duas linhas?. Supondo que tem um prisma com aresta de $10\,cm$, calcule o 
o declive mínimo para $\left(\frac{\ud n}{\ud \lambda} \right)$?
\end{enumerate}


\section{\sf Protocolo Experimental}


\begin{enumerate}
\item Ligue  a  lâmpada  espetral  e  espere  10  a  15  minutos    até  que  se  estabeleça  o 
equilíbrio térmico no seu interior. 

\end{enumerate}

%\newpage
\section*{\sf Apêndice}



\end{document} 