%%&program=xelatex
%&encoding=UTF-8 Unicode
% SVN keywords
% $Author$
% $Date$
% $Revision$
% $URL$
\documentclass[a4paper,12pt]{article}  % Comments after  % are ignored
%\usepackage{hyperref}                 % For creating hyperlinks in cross references
%
\usepackage{ifxetex}% for XELATEX, or PDFlatex
\usepackage{ifplatform} 
%
\ifxetex
	\usepackage{polyglossia} \setmainlanguage{portuges}
	\usepackage{fontspec}
	\ifwindows
		\setmainfont[Ligatures=TeX]{Garamond}
		\setsansfont[Ligatures=TeX]{Gill Sans MT}
		\setmonofont[Scale=0.95]{Courier}
	\fi
	\iflinux
		\setmainfont[Ligatures=TeX]{Linux Libertine O}
		\setsansfont[Ligatures=TeX,Scale=MatchLowercase]{Linux Biolinum}
		\setmonofont[Scale=MatchLowercase]{Courier}
	\fi
	\ifmacosx
	% add settings
	% Use xelatex -no-shell ...
	\fi
	\usepackage{xcolor,graphicx} 
\else
	\usepackage[portuguese]{babel}
	%\usepackage[latin1]{inputenc}
	\usepackage[utf8]{inputenc}
	\usepackage[T1]{fontenc}
	\usepackage{graphics}                 % Packages to allow inclusion of graphics
	\usepackage{color}                    % For creating coloured text and background
\fi

\usepackage{enumitem}
\setlist{nolistsep}

\usepackage{amsmath,amssymb,amsfonts} % Typical maths resource packages
\usepackage[retainorgcmds]{IEEEtrantools}
\usepackage{caption}
\usepackage[bookmarks,colorlinks]{hyperref}

\oddsidemargin 0cm
\evensidemargin 0cm

\pagestyle{myheadings}         % Option to put page headers
                               % Needed \documentclass[a4paper,twoside]{article}
\markboth{{\small \it  Laboratório de Física Experimental Básica}}
{{\small\it MEFT - 1º Sem. 2015/2016} }

\addtolength{\hoffset}{-0.5cm}
\addtolength{\textwidth}{2.5cm}
\addtolength{\topmargin}{-1.5cm}
\addtolength{\textheight}{3cm}

%\textwidth 15.5cm
%\topmargin -1.5cm
\setlength{\parindent}{0pt}
\setlength{\parskip}{1ex  plus  0.5ex  minus  0.2ex}
%\parindent 0.5cm
%\textheight 25cm
%\parskip 1mm


% Math macros
\newcommand{\ud}{\,\mathrm{d}} 
\newcommand{\HRule}{\rule{\linewidth}{0.5mm}}

\author{Prof. Bernardo B. Carvalho} 

%%%%, Bernardo Brotas Carvalho\\bernardo@ipfn.ist.utl.pt} 
\date{ Outubro 2012} 

\begin{document} 

	\includegraphics[width=0.2\textwidth]{../logo-ist}%\\[1cm]  %%  Logo_IST_color

	\HRule \\[0.5cm]
	{ \huge \sf  \textsc{Indice de Refração do vidro de um Prisma pelo método  do Desvio Mínimo.}} \\[0.4cm] % \bfseries 
%	{ \huge \sf  \textsc{Construções Geométricas em Lentes Delgadas (aproximação paraxial)} }\\[0.4cm] % \bfseries 
	{ \large \bfseries Poder Dispersivo e Poder de Resolução do Vidro.}\\
%	{ \large \bfseries Procedimento Experimental}\\
	\HRule \\%[0.5cm]

\section{\sf Princípio do método}
Um prisma de um meio transparente, homogéneo e isótropo de índice de refração, $n$, colocado no percurso de um feixe luminoso incidente produz um desvio angular, $\delta$, no feixe emergente que depende do ângulo de incidência, $i_1$. Pode provar-se que a função desvio angular $vs$ apresenta um ponto de estacionariedade (i.e., derivada nula) que é um mínimo se $n > 1$. 
Mostra-de também que nessa situação as direções dos dois feixes são igualmente inclinadas em relação às faces do prisma, i.e. quando o ângulo de incidência é igual ao ângulo de transmissão 
emergente (ver Apêndice). 
Nesse caso  (Figura \ref{fig:desvio}) o índice de refração, $n$, pode ser calculado simplesmente através da expressão seguinte: 

\begin{figure}[hb]  \centering 
	\includegraphics[width=0.6\textwidth]{desvio}
	\caption{Esquema da transmissão do feixe incidente num prisma colocado na plataforma do goniómetro de Babinet. A direção do feixe transmitido desvia-se do ângulo $\delta$ em relação ao feixe incidente. \label{fig:desvio}} 
\end{figure}

\begin{equation}
	\label{eq:desviomim}
	n= \frac{\sin \left( \frac{\alpha+ \delta_{min}}{2} \right) } {\sin \left(  \frac{\alpha}{2} \right)}  
\end{equation}

em que $\alpha$ e  $\delta_{min}$ são os ângulos, respetivamente, do prisma e o desvio mínimo referido. Este desvio mínimo depende do comprimento de onda da radiação incidente, $\lambda$, e por consequência $n$ depende de $\lambda$. Define-se \emph{poder dispersivo} dum material como a derivada de $n$ em ordem a $\lambda$ ,  e escreve-se  como $\left( \frac{\ud n}{\ud \lambda } \right)$. Como esta função não é constante, deve indicar-se o valor do poder dispersivo relativo a um determinado valor de comprimento de onda incidente, ou $\left( \frac{\ud n}{\ud \lambda } \right)_{\lambda_i}$ .

O poder separador ou poder de resolução de um dado instrumento óptico\footnote{Optical Resolution}, $R_\lambda = \frac{\lambda}{\Delta \lambda} $,  é a capacidade que possui de poder permitir que se observem separadamente dois comprimentos de onda muito próximos, afastados de $\Delta \lambda$, na vizinhança de um valor médio $\overline{\lambda}$. Esta grandeza é adimensional e quanto maior for o seu valor, melhor é a resolução do instrumento.

No caso de um prisma obtém-se para $R$ a seguinte expressão (ver Apêndice), se a fonte é linear e se dispõe paralelamente à aresta do prisma:
 \begin{equation}
	\label{eq:resolu}
	R_\lambda = l\,\left(\frac{\ud n}{\ud \lambda} \right)_\lambda 
\end{equation}

em que $l$ é o \emph{maior percurso} do feixe luminoso no interior do prisma.

Uma \emph{rede de difração}
% (ver apontamentos da Aula Teórica  “Interferência e Difração”)
permite também observar separadamente dois comprimentos de onda muito próximos.
No entanto para uma rede de difração linear a resolução além de variar com o comprimento de onda depende da \emph{ordem de difração}, $m$
 \begin{equation}
	\label{eq:resoludrifa}
	R_\lambda = m\,N 
\end{equation}
sendo $N$  o número de linhas da rede iluminadas pelo feixe.


\section{\sf A experiência}
\subsection{\sf Equipamento}

\begin{enumerate}
\item Goniómetro de Babinet.
\item Prisma triangular de vidro.
\item Lâmpada espetral de Mercúrio ou Hélio.
\end{enumerate}

No Trabalho anterior colocou-se o prisma na plataforma de modo a poderem observar-se as reflexões nas duas faces polidas. As direções dos raios refletidos fazem um ângulo $\theta$, que se mediu com o goniómetro e que é o dobro do ângulo principal do prisma $\alpha$.  Determinou-se facilmente assim o ângulo do prisma com uma precisão e exatidão muito melhores do que com um transferidor, por exemplo.

\begin{figure}[tb]  \centering 
	\includegraphics[width=0.6\textwidth]{angulo}
	\caption{Esquema da reflexão do feixe incidente num prisma colocado na plataforma do goniómetro de Babinet. As direções dos dois raios refletidos fazem entre si um ângulo $\theta$ que é o dobro do ângulo α do prisma. \label{fig:angulo}} 
\end{figure}

À partida podia determinar-se  o ângulo $\delta_{min}$ medindo com a objectiva a direção do raio incidente (sem prisma) e a direção do raio emergente que corresponda ao desvio mínimo (Figura \ref{fig:desvio}). No entanto para compensar as eventuais assimetrias do aparelho devem fazer-se observações para os desvios à esquerda e à direita.
% para os raios que emergem das duas faces que definem o ângulo do prisma. 
Neste caso é facil provar que o ângulo formado entre os dois raios emergentes (para a mesma côr) é o dobro do ângulo de desvio, $\delta_{min}$ (e não é necessário determinar a direção do raio incidente, $i_1$).
% segundo o ângulo $i_2$ em relação à normal (
% para cada comprimento de onda 


\begin{figure}[tb]  
\centering 
	\includegraphics[width=0.55\textwidth]{Espectrometro-Goniometro-S}
	\caption{Fotografia do goniómetro de Babinet utilizado na experiência. \label{fig:spectrometer}} 
\end{figure}


%\begin{figure}[!htb]  
%	\centering 
%	\includegraphics[width=0.55\textwidth]{babinet}
%	\caption{Esquema do Goniómetro de Babinet. Legenda: FL-fonte luminosa, C-colimador, F-fenda, Lc-lente convergente do colimador, Pt-plataforma, %P-Prisma, L-luneta, Obj-objetiva, Oc-ocular, Ret-retículo, N-nónio acoplado à luneta\label{fig:babinet}} 
%\end{figure}


O prisma que se usa é em geral de seção reta triangular (equilátero ou isósceles).
A lâmpada espetral é uma fonte de luz policromática  que contem dois elétrodos situados no interior de um invólucro (vidro em geral) onde existe uma substância “ativa”\footnote{Que dá o nome à lâmpada, e.g. Mercúrio, Hélio, Néon, etc.} em muito pequena quantidade, numa atmosfera rarefeita. A alimentação que em geral é de alta tensão e produz entre os elétrodos uma descarga que vaporiza, excita e ioniza a substância ativa. As diferentes excitações permitem transições radiativas que dão origem à emissão de um feixe constituido por diferentes comprimentos de onda bem definidos e que se encontram já muito bem identificados na literatura\footnote{Pode consultar a extensa base de dados online em  \href{http://physics.nist.gov/asd}{NIST Atomic Spectra Database}.}. 

Todas as lâmpadas espetrais emitem no ultravioleta que é nocivo para a pele e olhos dos observadores, mas o vidro no interior do qual se dá a descarga absorve a maior parte destas radiações perigosas. Para ainda reduzir os riscos, a lâmpada tem um invólucro em geral metálico apenas com uma abertura para permitir iluminar uma fenda estreita e regulável junto à luneta do goniómetro, que produz uma linha vertical na imagem.  A aresta do triangulo  definido pelas superfícies planas onde se produz a reflexão e/ou transmissão deve ficar paralela a esta fenda.

% ~
% \newpage
% \subsection{\sf Questões a responder ANTES da sessão de Laboratório:}
% \begin{enumerate}
% \item Descreva quais os objectivos do Trabalho que irá realizar na sessão de Laboratório. Indique as expressões que irá utilizar para obter as grandezas experimentais, bem como as expressões para calcular as incertezas. Inclua esta parte também no Relatório Impresso. Este irá constituir o ÚNICO meio de consulta na Prova Individual.
% \item Obtenha uma imagem típica da dispersão da Luz Branca num prisma triangular. A partir dessa figura pode concluir que índice de refração, $n(\lambda)$, é uma função crescente ou decrescente?
% \item Nessa figura de dispersão como pode identificar qual é a côr que está na posição de \emph{desvio mínimo}?
% \item Se na montagem de laboratório substituir a lâmpada de descarga pela luz Solar que imagem observaria na luneta do goniómetro?.
% \item O espectro de emissão do Hidrogéneo, na série de Balmer (transição eletrónica de nível $3 \to 2$) tem duas riscas no c.d.o. vermelho, respetivamente a $\lambda = 656.272\, nm$ e $\lambda = 656.2852\,nm$.
% Qual a Resolução mínima de um Instrumento (Espectrómetro) capaz de distinguir estas duas linhas?. Supondo que tem um prisma com aresta de $10\,cm$, calcule o
% o valor absoluto mínimo para $\left(\frac{\ud n}{\ud \lambda} \right)$?
% \end{enumerate}


\section{\sf Protocolo Experimental}

\begin{enumerate}
\item Ligue  a  lâmpada  espetral  e  espere  10  a  15  minutos    até  que  se  estabeleça  o 
equilíbrio térmico no seu interior. 
\item Enquanto espera comece por regular a ótica  do goniómetro tal como descrito no Guia do Trabalho anterior.
\item Verifique o nivelamento horizontal do goniómetro e da plataforma onde vai colocar o prisma com a ajuda de um nível de bolha. 
\item Utilize o valor do ângulo, $\alpha$, entre as faces polidas prisma obtido no trabalho anterior.
\item  Observe agora  a  transmissão  das várias côres  através  do  prisma  com  o  feixe  incidente  numa  das  faces, na posiçao da Figura 3 do guia do trabalho anterior.  Se o instrumento estiver bem focado deve  observar   uma  série  de 
imagens colorida da fenda, i.e. riscas isoladas, uma por cada côr (ou comprimento de onda $\lambda$).
\item Escolha duas dessas côres bem afastadas. Rodando o prisma obtenha um conjunto de valores que permita fazer um gráfico dos ângulos de desvio, $\delta(i_1)$, 
em função do ângulo de incidência, $i_1$ e um ajuste polinomial (Nesta fase não precisa de utilizar o nónio da escala de minutos). Pelo gráfico verifique que existe de facto um mínimo no ângulo de desvio,  $ \delta(i_{min})$, mas  que o ângulo $i_{min}$ e $\delta(i_{min})$ são  diferentes para cada côr.
\item  Em seguida, para medir o desvio mínimo,  \emph{para cada côr}, tem de se colocar o prisma para posição correspondente. Deve portanto olhar pela objectiva e fixando uma côr rodar o prisma até observar o desvio mínimo.
%Pode finalmente medir os ângulos de desvio, $\delta_{min(esq, dir)}(\lambda)$.
% agora com o nónio e o auxílio  do  parafuso  micrométrico  associado  à  plataforma.  
De seguida,  com  o  parafuso micrométrico associado à luneta, centrar  a imagem  da fenda no retículo e meça o  ângulo  $\delta_{min(esq)}(\lambda)$. Comece por usar todas as corês quando o prisma desvia para o lado esquerdo (cada elemento do grupo deve efectuar as suas medidas). 
\item Repita o ponto anterior, mas para o prisma a desviar o feixe para o lado direito. Calcule os desvio mínimos:
$$\delta_{min}(\lambda) = \frac{|\delta_{esq} -  \delta_{dir}|}{2}$$ 
\item Com os desvio mínimos e identificando  nas tabelas os valores  dos  comprimentos  de onda das riscas visíveis, calcule  
índice de refração a partir da Equ.~\ref{eq:desviomim} e represente  graficamente a função $n(\lambda)$. Ajuste uma curva polinomial aos pontos obtidos. Através da  estimativa da derivada desta função, calcule o poder dispersivo do vidro para o comprimento de onda médio, $\overline{\lambda}_{amare}$, das 
duas riscas amarelas do sódio ($\lambda \approx 589\,nm$). 
\item Faça uma estimativa da maior distância percorrida pelo feixe luminoso no prisma (que é aproximadamente o comprimento da aresta) e calcule 
  o  Poder  de  Resolução  do  prisma  para  o  $\overline{\lambda}_{amare}$, referido  no  ponto  anterior. 
%Compare este valor com o que obteria se usasse o mesmo feixe com uma rede de difração 
%de 600 linhas por milímetro. 
\item Substitua no centro da plataforma do goniómetro o prisma por uma rede de difração de 
600 linhas por milímetro. Compare a separação angular, $\Delta \delta_{rede}$, das duas riscas mais próximas, 
observadas  com  a  rede  com  a  que  obteve  para  as  mesmas  riscas,  usando  o  prisma. 
Comente, utilizando a expressão (\ref{eq:resoludrifa}). 

\end{enumerate}

%\newpage
\section*{\sf Apêndice}
\subsection*{\sf Ângulo do Prisma}

Considere-se a incidência nas condições da Figura \ref{fig:angulo}.
A figura plana quadrangular $ABEC$ tem dois ângulos retos $ABE$ e $ECA$. Assim o ângulo $CAB$, $\alpha$, é suplementar do ângulo $BEC$ e portanto o ângulo externo em $E$ (assinalado na figura) tem o mesmo valor do ângulo do prisma.
Na figura plana quadrangular DBEC está definido um ângulo $θ$ que é o ângulo formado pelas direções dos raios refletidos na face $AB$ e $AC$. 
Entre os ângulos de incidência e $θ$ existe a relação 

 \begin{equation}
	\label{eq:soma}
	2\, \beta  + 2 \, \gamma + \theta = 2 \, \pi
\end{equation}

e entre os ângulos de $DBEC$ 
 \begin{equation}
	\label{eq:soma2}
	\beta  +  \gamma + \theta  + \pi - \alpha= 2 \, \pi
\end{equation}

O sistema destas duas equações permite obter 
 \begin{equation}
	\label{eq:alpha}
	\alpha=  \theta /2 
\end{equation}

\subsection*{\sf Desvio mínimo}
Supondo um prisma que tem um índice de refração que em relação ao meio em que está imerso $n$, na configuração da Figura \ref{fig:desvio}, os feixes que são transmitidos através do mesmo sofrem um desvio $\delta(\lambda)$. Os ângulos  $\alpha$ e de desvio $\delta $  são exteriores respetivamente aos triângulos $BCD$ (em $D$) e $BEC$ (em $E$) e portanto 


\begin{IEEEeqnarray}{rCl}
\delta &  =  &  (i_1 - t_1) +  (t_2 - i_2) \\
\alpha &  =  &  t_1  + i_2 \label{eq:soma3}
\end{IEEEeqnarray}

o que permite obter 
 \begin{equation}
	\label{eq:delta}
	\delta   =    i_1  +  t_2  - \alpha 
\end{equation}

Estes ângulos satisfazem à lei de Snell-Descartes da transmissão.

 \begin{equation}
	\label{eq:snell}
	n = \frac{\sin i_1}{\sin t_1}  =  \frac{\sin t_2}{\sin i_2}  
\end{equation}

No caso geral o ângulo $\delta$ depende do ângulo de incidência $i_1$ e pode provar-se que a função envolvida tem uma estacionariedade. Para encontrar o \emph{desvio mínimo} $\delta_{min}$, calculemos  a derivada de $\delta$ em relação a $i_1$ (expressão \ref{eq:delta}).

 \begin{equation}
	\label{eq:deriv}
	\frac{\ud \delta}{\ud i_1}   =  1 + 	\frac{\ud t_2}{\ud i_1} 
\end{equation}

Obtendo $\sin i_1$ e $\sin t_2$ da relação (\ref{eq:snell}) e aplicando-lhe derivada em ordem a $i_1$ é-se conduzido a 
\begin{IEEEeqnarray}{rCl}
\cos i_1 &  =  & n \,\cos t_1 \cdot  \frac{\ud t_1}{\ud i_1} 	\label{eq:deriv2} \\
\cos t_2  \cdot   \frac{\ud t_2}{\ud i_1}  &  =  & n \, \cos i_2  \cdot  \frac{\ud i_2}{\ud i_1} 	\label{eq:deriv3}
\end{IEEEeqnarray}

Mas atendendo à relação (\ref{eq:soma3})
 \begin{equation}
	\label{eq:deriv4}
	\frac{\ud t_1}{\ud i_1}  = - \frac{\ud i_2}{\ud i_1} 
\end{equation}

e combinando (\ref{eq:deriv2}), (\ref{eq:deriv3}) e (\ref{eq:deriv4}) obtém-se
 \begin{equation}
	\label{eq:deriv5}
	\frac{\ud t_2}{\ud i_1}  = - \frac{\cos i_2\,\cos i_1}{\cos t_2\,\cos t_1} 
\end{equation}

e a relação (\ref{eq:deriv}) vem

\begin{equation}
	\label{eq:deriv6}
	\frac{\ud \delta}{\ud i_1}   = 1-  \frac{\cos i_2\,\cos i_1}{\cos t_2\,\cos t_1} 
\end{equation}
Esta função admite um zero para
\begin{equation}
	\label{eq:deriv_0}
	{\cos i_2\,\cos i_1}={\cos t_2\,\cos t_1} 
\end{equation}

Atendendo a (\ref{eq:snell}) e à relação entre coseno e seno obtém-se
\begin{equation}
	\label{eq:deriv_1}
	\sin^2 t_1 \cdot  (1 - n^2)  = \sin^2 i_2 \cdot  (1 - n^2)
\end{equation}

que para os ângulos considerados ($\le  \pi/2)$ e para $n\ne 1$ implica que 

\begin{IEEEeqnarray}{rCl}
t_1  &  =  & i_2 = t\\
i_1  &  =  & t_2 = i
\end{IEEEeqnarray}


Assim para $\frac{\ud \delta}{\ud i} =0 $  (que provaremos ser um mínimo)
\begin{equation}
	\label{eq:deltvmin}
	\delta_{min} = 2\,i- 2\,t = 2\,i - \alpha
\end{equation}

o que permite calcular $t$ a partir do ângulo do prisma ($t = \alpha / 2$) e $i$ a partir do ângulo de desvio mínimo e do ângulo do prisma ($i = (\alpha + \delta_{min} ) / 2$) e consequentemente obter a relação (\ref{eq:desviomim}) para o cálculo do índice de refração.

É necessário calcular  $\frac{\ud^2 \delta}{\ud i^2} =0 $ e verificar se é uma quantidade positiva ou negativa para os valores que anulam a primeira derivada com o objetivo de saber se a estacionariedade é um mínimo,  máximo  ou um ponto de inflexão.
Aplicando derivada em ordem a $i$ à expressão (\ref{eq:deriv6}) obtém-se

\begin{equation}
	\label{eq:d2eriv}
	\frac{\ud^2 \delta}{\ud i_1^2}   =   	\frac{\ud }{\ud i_1} \left(-  \frac{\cos i_2\,\cos i_1}{\cos t_2\,\cos t_1} \right)
\end{equation}

em que todos os argumentos das funções coseno dependem de $i_1$.
obtém-se 4 parcelas que são: 

\begin{IEEEeqnarray}{rCl}
 \frac{\cos i_1}{\cos t_2\,\cos t_1} \frac{\ud }{\ud i_1} \cos i_2 &  =  & 
 \frac{\cos i_1}{\cos t_2\,\cos t_1} (-\sin i_2) \frac{\ud i_2}{\ud i_1} 	\label{eq:d2eriv_a}\\
%
 \frac{\cos i_2}{\cos t_2\,\cos t_1} \frac{\ud }{\ud i_1} \cos i_1 &  =  & 
 \frac{\cos i_2}{\cos t_2\,\cos t_1} (-\sin i_1)  \label{eq:d2eriv_b} \\
 \frac{\cos i_2\,\cos i_1}{\cos t_1}  \frac{\ud }{\ud i_1} (\cos t_2)^{-1} &  =  & 
  \frac{\cos i_2\,\cos i_1}{\cos t_1} \sin t_2 (\cos t_2)^{-2} \frac{\ud t_2}{\ud i_1}\label{eq:d2eriv_c}\\
  \frac{\cos i_2\,\cos i_1}{\cos t_2}  \frac{\ud }{\ud i_1} (\cos t_1)^{-1}  &  =  & 
  \frac{\cos i_2\,\cos i_1}{\cos t_2} \sin t_1 (\cos t_1)^{-2} \frac{\ud t_1}{\ud i_1} \label{eq:d2eriv_d}
\end{IEEEeqnarray}

Atendendo a (\ref{eq:deriv2}) e (\ref{eq:deriv4}) substituidos em (\ref{eq:d2eriv_a}) e em (\ref{eq:d2eriv_c}) e a (\ref{eq:deriv5}) substituido em (\ref{eq:d2eriv_d}), as 4 parcelas conduzem respetivamente às expressões seguintes que são simplificadas quando se substitui $n$ (\ref{eq:snell}) e se impõem as condições que foram obtidas para o zero de   $\frac{\ud \delta}{\ud i_1} $ 

\begin{IEEEeqnarray}{rCl}
 \frac{\cos i_1}{\cos t_2\,\cos t_1} (-\sin i_2) \frac{\cos i_1}{n\, \cos t_1}  &  =  & 	
    \frac{\sin^2 t}{\cos^2 t}  \frac{\cos i}{\sin i} \\
%
 \cdots &  =  & -\frac{\sin i}{\cos i} \\
 \cdots &  =  & -\frac{\sin i}{\cos i} \\
 \cdots &  =  & 	  \frac{\sin^2 t}{\cos^2 t}  \frac{\cos i}{\sin i} 
\end{IEEEeqnarray}

Assim obtém-se 

\begin{equation}
	\label{eq:d22e}
	\frac{\ud^2 \delta}{\ud i_1^2}   =   -2 \tan^2 t  \, \frac{1}{\tan i} + 2\tan i = 2 \, \tan i 
	\left(1 - \frac{\tan^2 t}{\tan^2 i} \right)
\end{equation}
Esta expressão é positiva para $ \tan^2 t < \tan^2 i $ o que implica $t < i$ ($i, t \le \pi/2$), ie, para $n > 1$ e é negativa para $n < 1$. No caso do prisma de vidro imerso no ar $n > 1$ e portanto (\ref{eq:d22e}) será positiva o que confirma que a condição de estacionariedade corresponde a um mínimo.

\subsection*{\sf Poder de resolução do prisma}

A capacidade de observar separadamente dois comprimentos de onda muito próximos está relacionada com a variação do ângulo de desvio $\delta$ com o comprimento de onda λ que se designa por dispersão angular   
$\frac{\ud \delta}{\ud \lambda}$ e que depende do coeficiente de dispersão  $\frac{\ud n}{\ud \lambda}$ na forma  

\begin{equation}
	\label{eq:poder_res}
	\frac{\ud \delta}{\ud \lambda} =\frac{\ud \delta}{\ud n} \, \frac{\ud n}{\ud \lambda}
\end{equation}

Atendendo a (\ref{eq:delta}) $\frac{\ud \delta}{\ud n} = \frac{\ud t_2}{\ud n}$ (para $\alpha$ e $i_1$ constantes). Derivando a relação (\ref{eq:snell}) em ordem a $n$ obtém-se

\begin{IEEEeqnarray}{rCl}
\cos t_2 =  \frac{\ud t_2}{\ud n}  &  =  &  \sin t_2 + n\, \cos i_2  \frac{\ud i_2}{\ud n}\\
0 &  =  &  \sin t_1 +  n\, \cos t_1  \frac{\ud t_1}{\ud n}
\end{IEEEeqnarray}

Obtém-se assim 

\begin{IEEEeqnarray}{rCl}
\frac{\ud \delta}{\ud n}    &  =  &   \frac{\sin i_2}{\cos t_2} +  \frac{\sin t_1\; \cos i_2}{\cos t_2 \; \cos t_1 } =  \frac{\sin \alpha}{\cos t_2 \; \cos t_1 } \\
\frac{\ud \delta}{\ud \lambda}    &  =  & \frac{\ud n}{\ud \lambda} \frac{\sin \alpha}{\cos t_2 \; \cos t_1 } 
\end{IEEEeqnarray}

Considerando um feixe paralelo de largura $l_1$ como se indica na Figura \ref{fig:feixe}, que incide no prisma segundo um ângulo $i_1$ e que emerge segundo $t_2$ com largura $l_2$, fazendo um percurso máximo no prisma $l$, pode provar-se (os senos dos ângulos de um triângulo são diretamente proporcionais aos lados opostos) que 


\begin{figure}[!htb]  \centering 
	\includegraphics[width=0.6\textwidth]{feixe}
	\caption{Trajeto de um feixe luminoso paralelo num prisma. \label{fig:feixe}} 
\end{figure}

\begin{equation}
	\label{eq:sin_alpha}
	\frac{\sin \alpha}{l}= \frac{\sin (\pi/2 - t_1)} {AC}
\end{equation}

e atendendo a que $l_2= AC \cos t_2$  obtém-se que 

\begin{equation}
	\label{eq:delt_lamd}
	\frac{\ud \delta}{\ud \lambda}  =  \frac{\ud n}{\ud \lambda} \frac{l}{l_2}
\end{equation}

Assim uma pequena variação de comprimento de onda $\Delta \lambda$ produz uma variação do ângulo de desvio $\Delta \delta$  tal que 

\begin{equation}
	\label{eq:Delt_delta}
	\Delta \delta =  \frac{\ud n}{\ud \lambda} \frac{l}{l_2} \Delta \lambda 
\end{equation}

O critério de Rayleigh para que dois comprimentos de onda estejam resolvidos, ie possam ser detetados separadamente, é que o máximo de intensidade (de ordem $n \ge 1$) de um deles coincida com o mínimo de intensidade do outro (Figura \ref{fig:gauss})

\begin{figure}[ht]  \centering 
	\includegraphics[width=0.6\textwidth]{gauss}
	\caption{Critério de Rayleigh da resolução de duas riscas espetrais (a vermelho a soma da intensidade das riscas. \label{fig:gauss}} 
\end{figure}

Pelas leis de Difração o primeiro mínimo de intensidade da figura de difração de uma fenda de largura $l_2$ dista angularmente do máximo principal de:  $\sin \theta = \lambda/l_2$.
Para dois comprimentos de onda muito próximos, $\sin  \theta \approx \theta $, que neste caso é o desvio angular $\Delta \delta$. Assim $\Delta \delta= \lambda / l_2$ e obtém-se para a resolução do prisma: 

~\begin{equation}
	\label{eq:resolup}
	R_\lambda  =  \frac{\lambda}{\Delta \lambda} = l \left(\frac{\ud n}{\ud \lambda} \right )_\lambda 
\end{equation}



\end{document} 