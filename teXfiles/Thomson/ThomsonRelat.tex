%%&program=xelatex
%&encoding=UTF-8 Unicode
% SVN keywords
% $Author: bernardo $
% $Date: 2014-07-24 19:33:49 +0100 (Thu, 24 Jul 2014) $
% $Revision: 6559 $
% $U_aRL: http://metis.ipfn.ist.utl.pt/svn/cdaq/Users/Bernardo/Aulas/LFEB/teXfiles/Planck/Planck.tex $
\documentclass[a4paper,12pt]{article}  % Comments after  % are ignored
%\usepackage{hyperref}                 % For creating hyperlinks in cross references
%
\usepackage{ifxetex}% for XELATEX, or PDFlatex
\usepackage{ifplatform} 
%
\ifxetex
	\usepackage{polyglossia} \setmainlanguage{portuges}
	\usepackage{fontspec}
	\ifwindows
		\setmainfont[Ligatures=TeX]{Garamond}
		\setsansfont[Ligatures=TeX]{Gill Sans MT}
		\setmonofont{Consolas}
%		\setmonofont[Scale=MatchLowercase]{Courier}
		\setmonofont[Scale=0.95]{Courier}
	\fi
	\iflinux
		\setmainfont[Ligatures=TeX]{Linux Libertine O}
		\setsansfont[Ligatures=TeX,Scale=MatchLowercase]{Linux Biolinum}
		\setmonofont[Scale=MatchLowercase]{Courier}
	\fi
	\ifmacosx
	% add settings
	% Use xelatex -no-shell ...
		\setmainfont[Ligatures=TeX]{Garamond}
		\setsansfont[Ligatures=TeX]{Helvetica}
		\setmonofont{Consolas}
	\fi
	\usepackage{xcolor,graphicx} 
\else
	\usepackage[portuguese]{babel}
	%\usepackage[latin1]{inputenc}
	\usepackage[utf8]{inputenc}
	\usepackage[T1]{fontenc}
	\usepackage{graphics}                 % Packages to allow inclusion of graphics
	\usepackage{color}                    % For creating coloured text and background
\fi

\usepackage{enumitem}
\setlist{nolistsep}

\usepackage{amsmath,amssymb,amsfonts} % Typical maths resource packages
\usepackage[retainorgcmds]{IEEEtrantools}
\usepackage{caption}

\usepackage{tikz}
\usetikzlibrary{calc,arrows,decorations.pathmorphing,intersections}

\usepackage[font={small,sf},labelfont={bf},labelsep=endash]{caption}
\usepackage{sansmath}

\def\width{18}
\def\hauteur{11}

\oddsidemargin 0cm
\evensidemargin 0cm

\pagestyle{myheadings}         % Option to put page headers
                               % Needed \documentclass[a4paper,twoside]{article}
\markboth{{\small \it  Laboratório de Física Experimental Básica}}
{{\small\it IST - MEFT -LFEB - 1º Sem. 2015/2016} }

\addtolength{\hoffset}{-0.5cm}
\addtolength{\textwidth}{2.5cm}
\addtolength{\topmargin}{-1.5cm}
\addtolength{\textheight}{3cm}

%\textwidth 15.5cm
%\topmargin -1.5cm
\setlength{\parindent}{0pt}
\setlength{\parskip}{1ex  plus  0.5ex  minus  0.2ex}
%\parindent 0.5cm
%\textheight 25cm
%\parskip 1mm


% Math macros
\newcommand{\ud}{\,\mathrm{d}} 
\newcommand{\HRule}{\rule{\linewidth}{0.5mm}}

\author{Prof. Bernardo B. Carvalho} 

%%%%, Bernardo Brotas Carvalho\\bernardo.carvalho@tecnico.ulisboa.pt} 
\date{ Outubro 2015} 

\begin{document} 

%	\includegraphics[width=0.2\textwidth]{../logo-ist}%\\[1cm]  %%  Logo_IST_color

%	\HRule \\[0.5cm]
%	{ \huge \sf  \textsc{Construções Geométricas em Lentes Delgadas (aproximação paraxial)} }\\[0.4cm] % \bfseries 
%	{ \large \bfseries Determinação da constante de Planck.}\\

%	{ \large \bfseries Procedimento Experimental}\\
%	\HRule \\%[0.5cm]
%TURNO:________ GRUPO:________ DATA: ___ \\
%Número: _______ Nome: ________________ \\
%Número: _______ Nome: ________________ \\
{  \sf  Relatório da Experiência de Thomson} %[0.4cm] % \bfseries 
Turno:\underline{\makebox[1.5cm][l]{~}} Grupo:\underline{\makebox[1.5cm][l]{~}} Data:\underline{\makebox[1.5cm][l]{~}}\\
\noindent Número:~\underline{\makebox[2cm][r]{~}} Nome:~\underline{\makebox[10cm][r]{~}} \\
\noindent Número:~\underline{\makebox[2cm][r]{~}} Nome:~\underline{\makebox[10cm][r]{~}} \\
\noindent Número:~\underline{\makebox[2cm][r]{~}} Nome:~\underline{\makebox[10cm][r]{~}} 


\section{\sf Trabalho preparatório a realizar ANTES da sessão de Laboratório:}
\begin{enumerate}
\item Descreva quais os objectivos do Trabalho que irá realizar na sessão de Laboratório. 
\item Desenhe um diagrama dos campos elétricos, magnéticos, da velocidade do eletrões e forças aplicadas nas diferentes zonas do TRC.
\item Escolha os 5 pares de coordenadas, $(y,\, \pm z)$, na grelha do tubo TRC que irá utilizar nos ensaios de deflexão magnética, de modo a obter os maiores valores  de $R$ possíveis. Preencha as 4 primeiras colunas da Seção\ref{sec:dados}.
\end{enumerate}

\subsection{\sf Objectivos do Trabalho}
\noindent\underline{\makebox[\textwidth][r]{~}} \\
\noindent\underline{\makebox[\textwidth][r]{~}} \\
\noindent\underline{\makebox[\textwidth][r]{~}} \\
\noindent\underline{\makebox[\textwidth][r]{~}} \\
\noindent\underline{\makebox[\textwidth][r]{~}} \\
\noindent\underline{\makebox[\textwidth][r]{~}} \\
\noindent\underline{\makebox[\textwidth][r]{~}} \\

\subsubsection{\sf Equações }
Escreva no seguinte quadro todas as equações necessárias para calcular as grandezas bem com as suas incertezas.
\begin{center}
\framebox[13cm]{\rule{0pt}{6.5cm}}
\end{center}


\section{\sf Relatório}
\subsection{\sf DETERMINAÇÃO DE $q/m$ POR  DEFLEXÃO MAGNÉTICA}
\subsubsection{\sf Montagem Experimental}
Desenhe um diagrama da experiência. Inclua uma lista e legenda dos Instrumentos e a sua resolução e incerteza.
\begin{center}
\framebox[18cm]{\rule{0pt}{6.5cm}}
\end{center}



%\newpage

%\begin{figure}[!btp]
%     \makebox[\textwidth]{\framebox[18cm]{\rule{0pt}{7cm}}}
%     \caption{Montagem Experimental\label{montag}}
%\end{figure}

\subsubsection{\sf Medidas Experimentais e Cálculos Intermédios } \label{sec:dados}
Preencha as seguintes tabelas indicando  apenas os algarismos significativos. Poderá em alternativa utilizar folhas de cálculo, com o mesmo formato (apresentando-as em anexo) mas terá de peencher as colunas 1, 2, 3, 4 das tabelas seguintes e as colunas 1 e 6 das secção \ref{sec:calc}. Em qualquer dos casos terá que verificar as contas com auxílio calculadora, para um dos ensaios e na presença do docente.

\begin{center}
	\noindent	$U_a =$ \underline{\makebox[1.5cm][r]{~}} $V$ ,  $\quad \delta U_a=$	\underline{\makebox[1cm][r]{~}} $V$, $\quad \delta (y,z)=$ \underline{\makebox[1cm][r]{~}} mm  \\

	\renewcommand{\arraystretch}{0.75}
	\begin{tabular}{|c|c|c|c|c|c|c|c|}
	\hline
	y [cm]  & $z_+/z_-$ [cm]  & $R$ [m] & $\delta R$ [m] & $I_+$ [mA] & $I_-$ [mA] & $\overline{I}$   {\tiny $ = \frac{| I_+ + I_-|}{2}$ [mA] } & {\tiny $\delta I = \frac{| I_+ - I_-|}{2}$ [mA] } \\
	\hline
	 &  &  &  & &  &  & \\ \cline{5-8}
	 &  &  &  & &  & & \\ \cline{5-8}
	 &  &  &  & &  & & \\ \cline{5-8}
	 \hline
	 &  &  &  & &  & & \\ \cline{5-8}
	 &  &  &  & &  & & \\ \cline{5-8}
	 &  &  &  & &  & & \\ \cline{5-8}
	 \hline
	 &  &  &  & &  & & \\ \cline{5-8}
	 &  &  &  & &  & & \\ \cline{5-8}
	 &  &  &  & &  & & \\ \cline{5-8}
	 \hline
	 &  &  &  & &  & & \\ \cline{5-8}
	 &  &  &  & &  & & \\ \cline{5-8}
	 &  &  &  & &  & & \\ \cline{5-8}
	 \hline
	 &  &  &  & &  & & \\ \cline{5-8}
	 &  &  &  & &  & & \\ \cline{5-8}
	 &  &  &  & &  & & \\ \cline{5-8}
	 \hline
 	\end{tabular}
	
	\noindent	$U_a =$ \underline{\makebox[1.5cm][r]{~}}  $V$  %$\pm$  	\underline{\makebox[1cm][r]{~}}
	\begin{tabular}{|c|c|c|c|c|c|c|c|}
	\hline
	y [cm]  & $+z/-z$ [cm]  & $R$ [m] & $\delta R$ [m] & $I_+$ [mA] & $I_-$ [mA] & $\overline{I}$   {\tiny $\frac{| I_+ + I_-|}{2}$ [A] } & {\tiny $\delta I = \frac{| I_+ - I_-|}{2}$ [A] } \\
	\hline
	 &  &  &  & &  &  & \\ \cline{5-8}
	 &  &  &  & &  & & \\ \cline{5-8}
	 &  &  &  & &  & & \\ \cline{5-8}
	 \hline
	 &  &  &  & &  & & \\ \cline{5-8}
	 &  &  &  & &  & & \\ \cline{5-8}
	 &  &  &  & &  & & \\ \cline{5-8}
	 \hline
	 &  &  &  & &  & & \\ \cline{5-8}
	 &  &  &  & &  & & \\ \cline{5-8}
	 &  &  &  & &  & & \\ \cline{5-8}
	 \hline
	 &  &  &  & &  & & \\ \cline{5-8}
	 &  &  &  & &  & & \\ \cline{5-8}
	 &  &  &  & &  & & \\ \cline{5-8}
	 \hline
	 &  &  &  & &  & & \\ \cline{5-8}
	 &  &  &  & &  & & \\ \cline{5-8}
	 &  &  &  & &  & & \\ \cline{5-8}
	 \hline
 	\end{tabular}
	
	\noindent	$U_a =$ \underline{\makebox[1.5cm][r]{~}} $V$ 
	\begin{tabular}{|c|c|c|c|c|c|c|c|}
	\hline
	y [cm]  & $\pm z$ [cm]  & $R$ [m] & $\delta R$ [m] & $I_+$ [mA] & $I_-$ [mA] & $\overline{I}$   {\tiny $\frac{| I_+ + I_-|}{2}$ [A] } & {\tiny $\delta I = \frac{| I_+ - I_-|}{2}$ [A] } \\
	\hline
	 &  &  &  & &  &  & \\ \cline{5-8}
	 &  &  &  & &  & & \\ \cline{5-8}
	 &  &  &  & &  & & \\ \cline{5-8}
	 \hline
	 &  &  &  & &  & & \\ \cline{5-8}
	 &  &  &  & &  & & \\ \cline{5-8}
	 &  &  &  & &  & & \\ \cline{5-8}
	 \hline
	 &  &  &  & &  & & \\ \cline{5-8}
	 &  &  &  & &  & & \\ \cline{5-8}
	 &  &  &  & &  & & \\ \cline{5-8}
	 \hline
	 &  &  &  & &  & & \\ \cline{5-8}
	 &  &  &  & &  & & \\ \cline{5-8}
	 &  &  &  & &  & & \\ \cline{5-8}
	 \hline
	 &  &  &  & &  & & \\ \cline{5-8}
	 &  &  &  & &  & & \\ \cline{5-8}
	 &  &  &  & &  & & \\ \cline{5-8}
	 \hline
 	\end{tabular}	
\end{center}

%\begin{table}[!hbp]
%\begin{center}
%	\centering
%	\noindent	$U_a =$ \underline{\makebox[1.5cm][r]{~}} $\pm$  	\underline{\makebox[1cm][r]{~}} $V$ \\

%	\caption{Dados da Experiência de Deflexão Magnética} 
%	\label{tab:Dados}
%\end{table}


\subsubsection{\sf Cálculos de $q/m$}
\label{sec:calc}
%\vspace{1cm}
\begin{center}
	\begin{tabular}{l|c|c|c|c|c|c|}
	\cline{2-7}
	 & $\overline{I}\quad \pm$ [A] &  $B$ [mT] & $\delta B$  [mT] & $q/m$ [C/kg] & $\delta q/m$ [C/kg] & $\overline{q/m}\quad \pm \quad [\quad C/kg]$\\ \cline{2-7}
	 &&&&&&   \\ \cline{2-6}
	 &&&&&& \\ \cline{2-6}
	 $U_a=$ \underline{\makebox[1.2cm][r]{~}}$V$   &&&&&&  \\ \cline{2-6}
	 &&&&&& \\ \cline{2-6}
	 &&&&&& \\ \hline \hline
	  
	 &&&&&& \\ \cline{2-6}
	 &&&&&& \\ \cline{2-6}
	$U_a=$ \underline{\makebox[1.2cm][r]{~}}$V$  &&&&&& \\
	\cline{2-6}
	 &&&&&& \\ \cline{2-6}
	 &&&&&& \\
	 \hline \hline
	 &&&&&& \\ \cline{2-6}
	 &&&&&& \\ \cline{2-6}
	 $U_a=$ \underline{\makebox[1.2cm][r]{~}}$V$  &&&&&& \\ \cline{2-6}
	 &&&&&& \\ \cline{2-6}
	 &&&&&& \\
	 \cline{2-7} 
 	\end{tabular}
\end{center}


\subsubsection{\sf Resultados Finais. Explique os critérios que utilizou para obter as incertezas.}
\noindent  $q/m =$~(\underline{\makebox[1.5cm][r]{~}}$\pm$\underline{\makebox[1cm][r]{~}})$\times 10^{+} \;\;C/kg  $\\  

\noindent  Desvio à Exatidão $=$~\underline{\makebox[1cm][r]{~}}(\%), 
Incerteza relativa $=$~\underline{\makebox[1cm][r]{~}}($\%$) 

\subsection{\sf DETERMINAÇÃO DE $q/m$ POR DEFLEXÃO\\ MAGNÉTICA E ELÉTRICA QUASE COMPENSADAS }

\subsubsection{\sf Dados Experimentais e Cálculos}

%\begin{itemize}
%\item Preencha as tabelas seguintes:
%\end{itemize}
\begin{table}[!hbp]
%\begin{center}
	\centering
%	\noindent	$U_a =$ \underline{\makebox[1.5cm][r]{~}} $V$ \\ %$\pm$ \underline{\makebox[1cm][r]{~}} 
	\begin{tabular}{|c|c|c|c|c|c|}
	\hline
	 $U_a [V]$ & $I_{max}$ [mA] & $I_{min}$ [mA] & $\overline{I} \quad \pm \quad[mA]$ & $B \quad \pm \quad [mT]$   & $q/m \quad \pm \quad [C/kg] $ \\
	\hline
	 & &  &  & &   \\
 	\hline
 	 & &  &  & &   \\
	 \hline
	 
 	\end{tabular}\\[10pt]
	\noindent	$U_a =$ \underline{\makebox[1.5cm][r]{~}} $\pm$ \underline{\makebox[1cm][r]{~}} $V$ \\
	\begin{tabular}{|c|c|c|c|c|c|c|c|}
	\hline
	 $I_{max}$ [mA] & $I_{min}$ [mA] & $\overline{I}$ [A]	& $\delta I = \frac{| I_{max} - I_{min}|}{2}$ [A] & $B$ [T] & $\delta B$  [T] & $q/m$ [C/kg] & $\delta q/m$ [C/kg] \\
	\hline
	 &  &  & &  &  & & \\
	 \hline
 	\end{tabular}\\[10pt]
	\noindent	$U_a =$ \underline{\makebox[1.5cm][r]{~}} $\pm$ \underline{\makebox[1cm][r]{~}} $V$ \\
	\begin{tabular}{|c|c|c|c|c|c|c|c|}
	\hline
	 $I_{max}$ [mA] & $I_{min}$ [mA] & $\overline{I}$ [A]	& $\delta I = \frac{| I_{max} - I_{min}|}{2}$ [A] & $B$ [T] & $\delta B$  [T] & $q/m$ [C/kg] & $\delta q/m$ [C/kg] \\
	\hline
	 &  &  & &  &  & & \\
	 \hline
 	\end{tabular}
	
	\caption{Dados da Experiência 2} 
	\label{tab:Dados2}
\end{table}

\subsubsection{\sf Resultados}
\noindent  $q/m =$~(\underline{\makebox[1.5cm][r]{~}}$\pm$\underline{\makebox[1cm][r]{~}})$\times 10^{+} \;\;C/kg  $\\  


\noindent  Desvio à Exatidão $=$~\underline{\makebox[1cm][r]{~}}(\%), 
Incerteza relativa $=$~\underline{\makebox[1cm][r]{~}}($\%$) 


\subsection{\sf Trajetória não compensada}
Aumente agora o campo $B$ (com $I<= 3\; A$) de forma a visualisar uma trajetória não compensada.  Faça um esboço da curva observada, indicando os vetores das forças em jogo (com uma estimativa do seu valor em $N$), bem como as condições experimentais. Comente a figura obtida.
\begin{center}
\framebox[18cm]{\rule{0pt}{6.5cm}}
\end{center}

\subsection{\sf Análise e comparação dos dois métodos. Conclusões e Comentários Finais}
\noindent\underline{\makebox[\textwidth][r]{~}} \\
\noindent\underline{\makebox[\textwidth][r]{~}} \\
\noindent\underline{\makebox[\textwidth][r]{~}} \\
\noindent\underline{\makebox[\textwidth][r]{~}} \\
\noindent\underline{\makebox[\textwidth][r]{~}} \\
\noindent\underline{\makebox[\textwidth][r]{~}} \\
\noindent\underline{\makebox[\textwidth][r]{~}} \\
\noindent\underline{\makebox[\textwidth][r]{~}} \\
\noindent\underline{\makebox[\textwidth][r]{~}} \\
\noindent\underline{\makebox[\textwidth][r]{~}} \\
\noindent\underline{\makebox[\textwidth][r]{~}} \\
\noindent\underline{\makebox[\textwidth][r]{~}} \\
\noindent\underline{\makebox[17cm][r]{~}} \\
%\begin{center}
%     \makebox[\textwidth]{\framebox[18cm]{\rule{0pt}{5cm}}}
%     \caption{Montagem Experimental\label{montag}}
%\end{center}



%\newpage
%\section*{\sf Apêndice}



\end{document} 