%&program=xelatex
%&encoding=UTF-8 Unicode
% SVN keywords
% $Author$
% $Date$
% $Revision$
% $URL$
\documentclass[a4paper,twoside,12pt]{article}      % Comments after  % are ignored
%\usepackage{hyperref}                 % For creating hyperlinks in cross references
%
\usepackage{ifxetex}% for XELATEX, or PDFlatex
\usepackage{ifplatform} 
%
\ifxetex
	\usepackage{polyglossia} \setmainlanguage{portuges}
	\usepackage{fontspec}
	\ifwindows
		\setmainfont[Ligatures=TeX]{Garamond}
		\setsansfont[Ligatures=TeX]{Gill Sans MT}
		\setmonofont[Scale=MatchLowercase]{Courier}
	\fi
	\iflinux
		\setmainfont[Ligatures=TeX]{Linux Libertine O}
		\setsansfont[Ligatures=TeX,Scale=MatchLowercase]{Linux Biolinum}
		\setmonofont[Scale=MatchLowercase]{Courier}
	\fi
	\ifmacosx
	% add settings
	% Use xelatex -no-shell ...
	\fi
	\usepackage{xcolor,graphicx} 
\else
	\usepackage[portuguese]{babel}
	%\usepackage[latin1]{inputenc}
	\usepackage[utf8]{inputenc}
	\usepackage[T1]{fontenc}
	\usepackage{graphics}                 % Packages to allow inclusion of graphics
	\usepackage{color}                    % For creating coloured text and background
\fi


\usepackage{amsmath,amssymb,amsfonts} % Typical maths resource packages
\usepackage{multicol}

\oddsidemargin 0cm
\evensidemargin 0cm

\pagestyle{myheadings}         % Option to put page headers
                               % Needed \documentclass[a4paper,twoside]{article}
\markboth{{\small \sf  Laboratório de Física Experimental Básica}}
{{\small\it MEFT - 2013/2014} }

\textwidth 15.5cm
\topmargin -1cm
\parindent 0.5cm
\textheight 24cm
\parskip 1mm

\usepackage{enumitem}
\setlist{nolistsep}

% Math macros
\newcommand{\ud}{\,\mathrm{d}} 
\newcommand{\HRule}{\rule{\linewidth}{0.5mm}}

\author{Prof. Bernardo B. Carvalho} 

%%%%, Bernardo Brotas Carvalho\\bernardo@ipfn.ist.utl.pt} 
\date{ Setembro 2012} 

\begin{document} 

%\begin{multicols}{1}
	\includegraphics[width=0.2\textwidth]{../logo-ist}%\\[1cm]  %%  Logo_IST_color

		\HRule \\[0.5cm]
	{ \huge \sf  \textsc{Estimativa do Valor da Carga Elétrica de \\
		Gotículas de Óleo} }\\[0.4cm] % \bfseries 
	{ \large \bfseries  
Ação de um campo elétrico sobre gotículas de óleo eletrizadas num fluido não 
condutor. (Experiência de Millikan)}\\
%	{ \large \bfseries Protocolo Experimental}\\
	\HRule \\%[0.5cm]

%	\HRule \\[0.5cm]
%	{ \huge \sf  \textsc{  Experiência de Millikan} }\\[0.4cm] % \bfseries 
%	{ \large \bfseries Determinação Experimental da Carga $q$ do Eletrão }\\
%	{ \large \bfseries Procedimento Experimental}\\
%	\HRule \\%[0.5cm]

%\twocolumn
%\end{multicols}

\begin{multicols}{2}

\section{\sf OBJECTIVO DO TRABALHO}
Pretende-se com este trabalho determinar a carga elétrica de pequenas gotas de óleo, tendo como objectivo final mostrar que a carga elétrica não aparece com uma quantidade qualquer, mas sempre como um múltiplo de uma unidade fundamental que é a carga do protão ou do eletrão. Deste modo um corpo eletrizado apresenta um excesso de carga de um dos sinais, mas sempre de valor múltiplo da carga elementar $q_{p,e}=\pm 1.602176565(35)\cdot 10^{-19}\,C$.
Traduz-se este facto dizendo-se que a carga elétrica se \emph{quantifica}.

Dentro das várias experiências elaboradas para mostrar este facto, uma montagem clássica é a do físico americano Robert A. Millikan\footnote{Millikan recebeu o prémio Nobel da Física em 1923 pelos seus trabalhos sobre a determinação da carga do eletrão e efeito fotoelétrico } (1869-1953), também chamada experiência da gota de óleo.

\section{\sf INTRODUÇÃO TEÓRICA}
%\section{\sf }
%\subsection{\sf }
\subsection{\sf Corpo esférico em queda livre num fluido}
Um corpo de dimensões muito pequenas\footnote{Com Número de Reynolds $Re= \frac{\rho v L}{\eta}$ inferior a $\simeq 100$}  ao mover-se com uma velocidade relativamente baixa através de um fluido (líquido ou gás), fica sujeito a uma força de atrito aproximadamente proporcional à sua velocidade, através da expressão:

\begin{equation}
	\label{eq:f_atrito}
	\vec{F}_{at} = - k \, \eta \vec{v}
\end{equation}

em que $\eta$ é o coeficiente de viscosidade do fluido, $\vec{v}$ é a velocidade do corpo e $k$ é um coeficiente que depende da forma do corpo  que toma o valor: 
\begin{equation}
	\label{eq:coef_atrito}
	k = 6 \pi \, R
\end{equation}
no caso de o corpo ser uma esfera de raio R (lei de Stokes).

O coeficiente $k$ virá assim expresso em \emph{metro} no Sistema Internacional (SI) e o coeficiente de viscosidade em $Pa\cdot s$ (ou $kg/m\,s$).
Normalmente a unidade de viscosidade que aparece na literatura é a unidade do sistema C.G.S. ($g/cm\,s$) que é designada por Poise (abreviatura $P$), verificando-se então a equivalência:

\begin{equation*}
	1 \, P = 0.1\, Pa\cdot s
\end{equation*}

Quando um corpo de massa m, cai em queda livre sob a ação do seu peso ($\vec{P}=m\vec{g}$) através de um fluido, o seu movimento de queda será travado pela força de atrito e a equação do movimento escreve-se:

\begin{equation}
	\label{eq:mov}
	m\,a \equiv m\, \frac{\ud\, v}{\ud\, t} =  m\,g - k  \, \eta \, v
\end{equation}

Sendo o peso do corpo constante, a aceleração $a$ produz um aumento contínuo em $v(t)$ e um aumento na força de atrito. Desta forma para uma determinada velocidade limite $v_L$, o segundo membro de (\ref{eq:mov}) anula-se e o corpo passará a deslocar-se com movimento uniforme. A velocidade limite $v_L$, será então obtida fazendo na equação (\ref{eq:mov}), $a= 0$:

\begin{equation}
	\label{eq:vlimit}
	v_L = \frac{m\,g}{k  \, \eta}
\end{equation}

o que poderá ser facilmente constatado pela resolução da equação (\ref{eq:mov}), pois a sua solução é da forma:

\begin{equation}
	\label{eq:vlimita}
	v(t) = \frac{m\,g}{k  \, \eta} (1 - e^{- (k\,\eta / m) t})
\end{equation}

à qual corresponde à seguinte representação gráfica.

\end{multicols}

\begin{figure*}
	[tb]  \centering 
	\includegraphics[width=0.8\textwidth]{./plote}
	\caption{ Evolução da velocidade de um corpo em queda livre sujeito a uma força de atrito. \label{fig:vLim}} 
\end{figure*}

\begin{multicols}{2}

e quando $t \to \infty :\; v(t) \to v_L = \frac{m\,g}{k  \, \eta} $

Se pretendermos ser mais rigorosos devemos substituir  em (\ref{eq:vlimit}) o peso do corpo pelo seu “peso aparente” no fluido. Isto é, quando um corpo cai em queda livre através de um fluido, sofre além da ação da força de atrito outra força de baixo para cima devida ao princípio de Arquimedes, igual ao peso do fluido deslocado pelo corpo. Então o peso real do corpo deverá ser substituído pelo seu “peso aparente” no fluido e as equações (\ref{eq:mov}) e (\ref{eq:vlimit}) deverão ser modificadas para:

\begin{equation}
	\label{eq:mov2}
	m\,a = m\,g - m_f\,g  - k  \, \eta \, v
\end{equation}


\begin{equation}
	\label{eq:vlimit2}
	v_L = \frac{(m - m_f)\,g}{k  \, \eta}
\end{equation}

Sendo $m_f$ a massa do fluido deslocado.

No caso de um corpo esférico de raio R, introduzindo a equação (\ref{eq:coef_atrito}) em (\ref{eq:vlimit2}) e atendendo a que:
\begin{equation*}
	m = \frac{4}{3} \pi R^3 \rho \quad \textrm{  e } \quad  m_f = \frac{4}{3} \pi R^3 \rho_f
\end{equation*}

Obtemos:
\begin{equation}
	\label{eq:vlimit3}
	v_L = \frac{2\,R^2\, (\rho - \rho_f)\,g}{9  \, \eta}
\end{equation}

em que $\rho$  e $\rho_f$ são as massas específicas do corpo e do fluido.

%

\end{multicols}

\begin{figure}
	[tb]  \centering 
	\includegraphics[width=0.5\textwidth]{./F_equil}
	\caption{Equilíbrio de forças elétrica e gravítica \label{fig:f_equil}} 
\end{figure}

\begin{multicols}{2}

\subsection{\sf Equilíbrio dum corpo carregado eletricamente, imerso num fluido através de um campo eletrico vertical.}

Seja o esquema representado na figura \ref{fig:f_equil}, em que entre duas placas condutoras paralelas se encontra um fluido não condutor. Aplica-se uma diferença de potencial $\Delta V = V_1 -V_1 > 0$ com a polaridade indicada na figura. Esta diferença de potencial criará um campo elétrico de baixo para cima. No caso de entre as duas placas se encontrar uma partícula de massa, $m$, e de carga, $q$, positiva\footnote{No caso da partícula estar carregada negativamente obteríamos o mesmo resultado invertendo o sentido do campo elétrico.} esta ficará sujeita a uma força elétrica que contrariará a sua queda.

No caso de entre as placas se poder considerar o campo elétrico, $\vec{E}$, como uniforme, o seu módulo será dado por:


\begin{equation*}
	E = \frac{\Delta\, V}{d}
\end{equation*}
sendo $d$ a distância entre as placas. O módulo da força elétrica que atua a partícula será então  dado por: 

\begin{equation*}
	F = |q| \frac{\Delta V}{d}
\end{equation*}

Deste modo a queda da partícula será agora contrariada, pela força de atrito e pela força elétrica. A equação (\ref{eq:mov2}) passa a escrever-se:

\begin{equation}
	\label{eq:mov3}
	m\,a = (m - m_f)\,g  - q \frac{\Delta\, V}{d} - k  \, \eta_{ar} \, v
\end{equation}

Variando a diferença de potencial(ddp), $\Delta V$, pode-se estabelecer o equilíbrio entre o peso da partícula e a força elétrica, conseguindo-se a sua paragem entre as placas. Nesse caso será $a=0$ e $v=0$ e tem-se:

\begin{equation}
	\label{eq:equil}
	0 = (m - m_f)\,g  - q \frac{\Delta\, V}{d} 
\end{equation}

Substituindo em (\ref{eq:equil}) $(m - m_f)\,g$   pela equação (\ref{eq:vlimit2}) obtemos:

\begin{equation*}
	v_L\, k\, \eta_{ar} = q \frac{\Delta\, V}{d}
\end{equation*}

E entrando também com (\ref{eq:coef_atrito}) no caso de a partícula ser esférica, obtemos:

\begin{equation}
	\label{eq:carga}
	q = \frac{6 \pi \, R \, \eta_{ar} \, d\, v_L}{\Delta V}  
\end{equation}

Onde

\begin{itemize}
\item $v_L$, a velocidade limite de queda da partícula através do fluido, na ausência do campo elétrico. 
\item $\eta_{ar} = 18.52 \cdot 10^{-5} P =  18.52 \cdot 10^{-6} \; Pa\cdot s $ (Viscosidade do ar a 23ºC).
\item $\rho = 973 \, kg/m^{3}$ (Massa específica do óleo de silicone).
\item $\rho_f = 1 \, kg/m^{3}$ (Massa específica do ar).
\item $g=9.800\, m/s^{2}$ (Aceleração gravítica em Lisboa).
\item $d=5.0\, mm$ (Distância entre placas).
\end{itemize}

\subsection{\sf Correções.}
\subsubsection{\sf Temperatura Ambiente.}

O valor da densidade da viscosidade do ar, no caso da temperatura ambiente se afastar muito de 23 ºC, terá de ser corrigido\footnote{Utilize por exemplo a caculadora \emph{online}: http://www.lmnoeng.com/Flow/GasViscosity.htm}

\subsubsection{\sf Dimensão das gotas.}

A Lei de Stokes não é exata quando as dimensões dos corpos esféricos forem comparáveis às distâncias entre as moléculas do fluido (ar).
Nestas condições Millikan verificou que a viscosidade $\eta_{ar}$ deveria ser substituída por

\begin{equation}
	\label{eq:correcao}
	\eta_{ar}' = \frac{\eta_{ar}}{1 + b/(p\,R)}  
\end{equation}

em que $b$ é uma constante igual $0.000617$, $p$ é pressão expressa em $cm$ de mercúrio\footnote{$1\,atm  = 1.013 \times 10^5 \,Pa = 1013 \, mbar = 76\, cm_{Hg}$}  e $R$ é o raio da gota em $cm$.

O valor corrigido $q_c$ pode ser determinado afetando o valor experimental $q$ por

\begin{equation}
	\label{eq:correcao1}
	q ' = q\, \left(\frac{\eta_{ar}'}{\eta_{ar}}\right)^{3/2}  =q\, \left(\frac{1}{1 + b/(p\,R)}\right)^{3/2}  
\end{equation}

\newpage
\section{\sf Procedimento Experimental}

\subsection{\sf Questões a responder ANTES da sessão de Laboratório:}
\begin{enumerate}
\item Para uma gota de chuva de diâmetro $R = 250\,\mu m$\footnote{Para gotas muito maiores a Lei de Stokes deixa de ser válida, pois a força de atrito passa a ser proporcional ao \emph{quadrado} da velocidade.}, $\rho_{H_2 O} = 1000 \, kg/m^{3}$, calcule a sua velocidade limite, a partir da espressão (\ref{eq:vlimit3}). Calcule o tempo característico, $\tau=m/k\,\eta_{ar}$, semelhante ao tempo necessário para alcançar a velocidade limite.
\item Qual a carga que a Terra teria de ter para que uma gota de óleo de $R = 10\,\mu m$, carregada com uma carga elementar positiva flutuar na atmosfera?
\item Pode o campo elétrico provocado pelas placas da motagem utilizada no laboratório ser considerado \emph{constante}?
\end{enumerate}

\subsection{\sf Material utilizado}

\begin{enumerate}
	\item Célula de Millikan com gerador de alta tensão (DC) regulável (potenciómetro) 
	\item  Atomizador e óleo de silicone 
	\item Multímetro, Cronómetro 
	\item Nível de bolha de ar. Parafuso para calibração do retículo do microscópio
\end{enumerate}
\section{\sf Procedimento experimental}

\begin{enumerate}
\item   Depois de verificar que a célula está horizontal e colocando o potenciómetro que 
controla a alimentação das placas do condensador no valor mínimo de tensão elétrica,
focalize o microscópio. Observe se o orifício que as gotículas atravessam está 
desobstruído.
\item    Verifique se o interruptor de inversão da alimentação do condensador está na posição ``Neutra''.
Rode o potenciómetro para uma posição que permita, quando ligar o interruptor de inversão, estabelecer um campo elétrico entre as placas do condensador. 
\item     Utilizando o pulverizador junto do orifício da célula produza uma pequena ``nuvem'' %“nuvem”
de gotículas de óleo. Observe através do microscópio o movimento das gotículas em 
frente do retículo (Note que a imagem encontra-se verticalmente invertida).
\item     Ligando o interruptor de inversão e variando a intensidade do campo elétrico e/ou o seu entido aplicado ao condensador por meio do 
potenciómetro verifique se as gotículas estão eletrizadas. Escolha uma das gotas e tente pará-la.
 \item Obrigue a gota a colocar-se numa determinada divisão do retículo, imobilizando-a. 
Leia o valor da diferença de potencial que permitiu essa imobilização. Anule o 
campo elétrico aplicado e verá a gota movimentar-se (com velocidade limite). Com um 
cronómetro meça o tempo necessário para que a gota faça um percurso de $N>4$ divisões
do retículo. Repita pelo menos duas vezes. Repita este processo para várias gotas (pelo 
menos cinco). 
\item   Calcule a velocidade limite média de cada gota e respetiva incerteza. Estime o raio e 
a carga de cada gota e correspondentes incertezas. Verifique se se pode considerar 
que a gotícula se move com a velocidade limite.
\item Calcule a carga de cada gota. Apenas para a gota de menor raio, calcule a carga corrigida
\item   Compare os valores das cargas média das gotas de menor carga  e a sua incerteza com o valor  tabelado da carga do eletrão. Discuta os resultados.
\end{enumerate}

Critique a precisão dos resultados obtidos. A partir dos resultados obtidos e atendendo
aos erros experimentais é possível, ou não, concluir sobre a quantificação da carga elétrica?

\end{multicols}

\end{document} 