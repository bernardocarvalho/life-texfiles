%%&program=xelatex
%&encoding=UTF-8 Unicode
% SVN keywords
% $Author$
% $Date$
% $Revision$
% $URL$
\documentclass[a4paper,12pt]{article}      % Comments after  % are ignored
%\usepackage{hyperref}                 % For creating hyperlinks in cross references
%
\usepackage{ifxetex}% for XELATEX, or PDFlatex
\usepackage{ifplatform} 
%
\ifxetex
	\usepackage{polyglossia} \setmainlanguage{portuges}
	\usepackage{fontspec}
	\ifwindows
		\setmainfont[Ligatures=TeX]{Garamond}
		\setsansfont[Ligatures=TeX]{Gill Sans MT}
		\setmonofont{Consolas}		
%		\setmonofont[Scale=MatchLowercase]{Courier}
	\fi
	\iflinux
		\setmainfont[Ligatures=TeX]{Linux Libertine O}
		\setsansfont[Ligatures=TeX,Scale=MatchLowercase]{Linux Biolinum}
		\setmonofont[Scale=MatchLowercase]{Courier}
	\fi
	\ifmacosx
	% add settings
	% Use xelatex -no-shell ...
		\setmainfont[Ligatures=TeX]{Garamond}
		\setsansfont[Ligatures=TeX]{Helvetica}
		\setmonofont{Consolas}
	\fi
	\usepackage{xcolor,graphicx} 
\else
	\usepackage[portuguese]{babel}
	%\usepackage[latin1]{inputenc}
	\usepackage[utf8]{inputenc}
	\usepackage[T1]{fontenc}
	\usepackage{graphics}                 % Packages to allow inclusion of graphics
	\usepackage{color}                    % For creating coloured text and background
\fi

\usepackage{enumitem}
\setlist{nolistsep}
\usepackage{amsmath,amssymb,amsfonts} % Typical maths resource packages

\oddsidemargin 0cm
\evensidemargin 0cm

\pagestyle{myheadings}         % Option to put page headers
                               % Needed \documentclass[a4paper,twoside]{article}
\markboth{{\small \it  Laboratório de Física Experimental Básica}}
{{\small\it MEFT - LFEB 1º Sem. 2014/2015} }

\addtolength{\hoffset}{-0.5cm}
\addtolength{\textwidth}{2.5cm}
\addtolength{\topmargin}{-1.5cm}
\addtolength{\textheight}{3cm}

%\textwidth 15.5cm
%\topmargin -1.5cm
\setlength{\parindent}{0pt}
\setlength{\parskip}{1ex  plus  0.5ex  minus  0.2ex}
%\parindent 0.5cm
%\textheight 25cm
%\parskip 1mm


% Math macros
\newcommand{\ud}{\,\mathrm{d}} 
\newcommand{\HRule}{\rule{\linewidth}{0.5mm}}

\author{Prof. Bernardo B. Carvalho} 

%%%%, Bernardo Brotas Carvalho\\bernardo@ipfn.ist.utl.pt} 
\date{ Outubro 2012} 

\begin{document} 

	\includegraphics[width=0.2\textwidth]{../logo-ist}%\\[1cm]  %%  Logo_IST_color

	\HRule \\[0.5cm]
	{ \huge \sf  \textsc{Medição da Velocidade da Luz} }\\[0.4cm] % \bfseries 
	{ \large \bfseries em diferentes Materiais homogéneos e isotrópicos}\\
%	{ \large \bfseries Procedimento Experimental}\\
	\HRule \\%[0.5cm]


\section{\sf Introdução}
Em muitas das experiências descritas na literatura para
determinacão da velocidade da luz foram utilizados feixes luminosos
pulsados (ou modulados), que percorrem determinados trajetos de maior ou menor
comprimento (Fig. \ref{fig:Fizeau}). 
No presente trabalho utiliza-se como fonte luminosa um díodo que
emite radiação eletromagnética visível com um comprimento de onda na
zona do vermelho. A tensão aplicada ao díodo emissor é alternada
sinusoidal de frequência $f_{mod}=50\,MHz$ e  assim a intensidade sinal luminoso
 emitido é \emph{Modulado em Amplitute} (AM), naquela frequência. 

\begin{equation*}
	\label{eq:f_am}
		s_{diodo}(t) = A(t) \cdot \sin ( 2\pi \cdot f_{luz} \, t) = \underbrace{A_0 \sin ( 2\pi \cdot 50\times 10^6 \, t)}_\text{Amplitude Modulada} \cdot \sin ( 2\pi \cdot f_{luz} \, t)
\end{equation*}

\begin{figure}
	[ht!b]  \centering 
	\includegraphics[width=0.5\textwidth]{Fizeau}
	\caption{Esquema do aparelho para determinar a Velocidade da Luz utilizado por Fizeau em 1849. \label{fig:Fizeau}} 
\end{figure}

%\section{\sf Introdução}

\section{\sf Base do Método}
No presente trabalho o feixe luminoso proveniente do díodo emissor
é forçado a percorrer um determinado trajeto sendo em seguida detetado
por um foto-díodo receptor (Fig. \ref{fig:Montagem}). O sistema de espelhos $E_1\,E_1$, pode deslocar-se ao longo de uma calha graduada,
sendo assim possível variar o comprimento do trajeto do raio luminoso. Os
sinais de amplitude impostos ao emissor e captados no receptor são aplicados\footnote{Depois de uma deteção \emph{heterodínica}, em que a frequência modulada é desviada \\ 
$f_{bat}=50\, MHz -50.050\, MHz = 50\,kHz$. Esta operação permite a utilização de um osciloscópio de banda de frequências mais reduzida.} 
aos canais de um osciloscópio funcionando em modo “XY”.

O sinal visualizado no osciloscópio é uma figura de Lissajous e como em ambos os sinais vertical e horizontal a frequência é a 
mesma\footnote{Na realidade são sinais \emph{coerentes}, pois provêm da mesma fonte}, é uma elipse com a equação geral:
\begin{equation}
	\label{eq:elipse}
	\sin^2 \delta = \frac{A_x^2}{A_{x0}^2} + \frac{A_y^2}{A_{y0}^2} - \frac{2 A_y\,A_x}{A_{y0}\,A_{x0}} \cos  \delta
\end{equation}

\begin{figure}
	[htb]  \centering 
	\includegraphics[width=0.8\textwidth]{Vel_esquema}
	\caption{Montagem no Laboratório para determinar a Velocidade da Luz. \label{fig:Montagem}} 
\end{figure}

Sendo $A_x(t)$ o sinal de intensidade captado no emissor,  $A_{x0}$ a sua amplitude máxima,  $A_y(t)$ o sinal
proveniente do recetor, $A_{y0}$ a sua amplitude máxima e $\delta$ a desfasagem entre os dois
sinais. O ângulo $\delta$ irá variar com a trajecto do raio luminoso.  A elipse pode degenerar em retas quando os dois sinais estiverem
em fase, $\delta = 2n\,\pi$ (nos quadrantes ímpares)  ou em oposição de fase, $\delta = (2n+1)\,\pi$ (nos quadrantes pares). 

\subsection{\sf Velocidade da Luz no ar}
Neste trabalho a
velocidade da luz é calculada a partir da determinação do comprimento do
caminho suplementar, ($2\,\Delta\,z$), que a luz tem de percorrer para que se passe de uma
situação em que os sinais estão em oposição de fase à situação contígua de
fase. Assim a luz percorre esse comprimento suplementar num intervalo de
tempo, ($\Delta t$) ,  que é igual a metade do periodo ($T/2$) do sinal modulante ($T=1/50\,MHz= 20\,ns$). 

\begin{figure}
	[htb]  \centering 
	\includegraphics[width=0.8\textwidth]{osci_fase}
	\caption{Figuras de Lissajous observadas no osciloscópio. \label{fig:fase}} 
\end{figure}

No ar teremos:
\begin{equation}
	\label{eq:vc}
	c_{ar} = \frac{2\,\Delta z_0}{T/2} 
\end{equation}

\subsection{\sf Velocidade da Luz em meios sólidos e líquidos}
 O índice de refração de a um meio material $1$ em relação a um meio material $0$
 para um dado comprimento de onda pode ser definido \footnote{Esta definição só é válida se as condutividades dos meios $0$ e $1$ forem nulas, ou seja nos \emph{dielétricos} perfeitos.}
 como a relação entre as velocidades de propagação da luz nos meios $0$ e $1$:

 \begin{equation}
	\label{eq:index}
	n_1 \equiv \frac{c_0}{c_1}  = \frac{\frac{1}{\sqrt{\varepsilon_0 \, \mu_0}} }{\frac{1}{\sqrt{\varepsilon_1 \, \mu_1}} } =
		\sqrt{\frac{\varepsilon_1 \, \mu_1}{\varepsilon_0 \, \mu_0}} = \sqrt{\varepsilon_r \, \mu_r}
\end{equation}

Nesta expressão $\varepsilon_0$, $\varepsilon_1$ e $\varepsilon_r$   e  $\mu_0$, $\mu_1$, e $\mu_r$ são as constantes dielétricas  as permeabilidades magnéticas respetivamente do meios $0$ e $1$ e a relativa $(\varepsilon_1= \varepsilon_r\, \varepsilon_0)$.

\begin{figure}[h!tb]  
	\centering 
	\includegraphics[width=0.8\textwidth]{Vel_esquema_bloco}
	\caption{Montagem para determinar índices de refração em sólidos e líquidos. \label{fig:Montagem_bloco}} 
\end{figure}

Se no percurso do feixe luminoso interpormos um bloco de material sólido ou líquido transparente 
 de comprimento $l_B$ (Figura \ref{fig:Montagem_bloco}) então a expressão  (\ref{eq:vc}) pode ser generalizada em:

\begin{equation}
	\label{eq:vc_bloco}
	{T/2}  = \frac{2\,\Delta z_1 - l_B}{c_{ar}}  +  \frac{l_B}{c_{B}}
\end{equation}
De onde se pode pode extrair a velocidade $c_{B}$. 

Pode obter-se directamente o indíce de refração $n_{B}$, a partir de (\ref{eq:vc}) e (\ref{eq:index}).

\begin{align}
	\label{eq:n_bloco}
	{T/2}  = \frac{2\,\Delta z_0}{c_{ar}}  &=  \frac{2\,\Delta z_1 }{c_{ar}} -   \frac{l_B}{c_{ar}}  +  \frac{l_B}{c_{B}} \nonumber \\ 
	\frac{2\,(\Delta z_0- \Delta z_1 )}{c_{ar}}  &= -   \frac{l_B}{c_{ar}}  +  \frac{l_B}{c_{B}} \nonumber \\
	\frac{2\,(\Delta z_0- \Delta z_1 )}{l_B} &= -1 +  \frac{c_{ar}}{c_{B}} \nonumber \\
	n_{B} &= 1 +  \frac{2\,(\Delta z_0- \Delta z_1 )}{l_B} 
\end{align}

 Neste trabalho será  determınada os índices de refração, e a velocidade da
luz, em  mais dois meios materiais, a resina acrílica e a água. 
 

\subsubsection{\sf Questões a responder ANTES da sessão de Laboratório:}
\begin{enumerate}
\item Descreva por palavras suas quais os objectivos do Trabalho que irá realizar na sessão de Laboratório (uma folha A4). Indique as expressões que irá utilizar para obter as grandezas experimentais, bem como as expressões para calcular as incertezas. Inclua esta parte também no Relatório. Este irá constituir o ÚNICO meio de consulta na Prova Individual.
\item A luz emitida pelo díodo vermelho tem um comprimento de onda de $\lambda \approx 600 \, nm$. Calcule $f_{luz}$ e compare com a frequência da modulação da amplitude.
\item Explique porque razão nesta experiência a luz utilizada  tem de ter a intensidade variável.
%, tal como na experiência da Velocidade do Som.
\item Uma outra montagem desenvolvida no IST utiliza um díodo laser de longo alcance. No entanto o máximo que se consegue modular a amplitude é, $f_{mod}=1\, MHz$. Calcule a distância
dos espelhos necessária para se obter o efeito de oposição de fase entre a amplitude dos sinais emitido e o recebido.
\end{enumerate}

\newpage
\section{\sf Protocolo Experimental}
\subsection{\sf Material utilizado}

\begin{enumerate}
\setlength{\itemsep}{0mm}
\item Unidade de emissão de luz (díodo emissor) e de recepção (díodo receptor)
(amplitude modulada por um sinal de frequência $50\,MHz$).
\item Duas lentes plano-cilíndricas.
\item Conjunto de dois espelhos planos para inversão do sentido de propagação da luz.
\item Calha graduada.  
\item Bloco de vidro acrílico transparente.
\item Dois tubos com cerca de 1 metro de comprimento para conter água ou ar. 
\item Osciloscópio de dois canais a funcionar em modo “XY”.
\end{enumerate}

\subsection{\sf Procedimento Experimental}
\subsubsection{\sf Regulação da montagem}
 
Comece por verificar se a montagem ótica está devidamente alinhada. Teste a evolução da forma da mancha 
luminosa ao longo do percurso óptico usando um alvo difusor (por exemplo, papel vegetal). 
Verifique que a mancha está centrada com  orifício que permite iluminar o 
díodo receptor. Coloque a lente no lado do díodo receptor.
Ajuste os parafusos dos espelhos de forma a maximizar a amplitude do sinal 
recebido.
Coloque os espelhos na posição zero da escala e, rodando o botão que modifica 
continuamente a diferença de fase entre os dois sinais, obtenha uma reta dos quadrantes ímpares. 

\subsubsection{\sf Velocidade de propagação da luz no ar}
Desloque os espelhos sobre a calha e observe a modificação da figura no ecrã do 
osciloscópio, em particular, nas posições que correspondem aos sinais recebidos estarem 
desfasados de $\pi/2$ (quadratura) e em oposição de fase. Para esta última situação registe a 
posição dos espelhos e o intervalo, $e_z$,  para o qual os sinais parecem estar ainda em 
oposição de fase. Repita a medida pelo menos duas vezes (cada observador). 
Calcule a velocidade de propagação da luz no ar, a sua incerteza e o desvios à exatidão\footnote{O valor $c_{ar}$ é muito próximo de $c_{vacuo} = 299 792 458\,m/s$, que é uma constante exacta do Sistema Internacional de Medidas. O índice de refração do ar para a luz visível é $n_{ar}=1.000293$}. 

\subsubsection{\sf Velocidade de propagação da luz no vidro acrílico}
Verifique de novo se no zero da escala de posição os sinais estão em fase. Coloque o bloco de vidro 
acrílico no feixe incidente de modo que entre perpendicularmente às faces e faça o percurso 
mais longo no vidro acrílico. Meça a posição e a incerteza correspondente dos espelhos para que 
que os dois sinais detetados estejam em oposição em fase. Repita a medida pelo menos 
duas vezes (cada observador).
Calcule o índice de refração obtido para o vidro acrílico, $n_{vidro}$, pela expressão (\ref{eq:n_bloco}),  e a sua incerteza. Calcule o valor da velocidade.

\subsubsection{\sf Velocidade de propagação da luz na água}
Verifique de novo que no zero da escala do espaço os sinais estão em fase. Coloque o tubo vazio de 
água nos suportes de modo que o feixe incidente entre perpendicularmente às faces.
Registe a posição dos espelhos que produz um sinal em oposição de fase com o incidente. 
Repita estas medidas para o tubo cheio de água. 
Calcule o valor da índice de refração obtida para a água $n_{agua}$, a sua incerteza e os desvio à exatidão\footnote{O valor tabelado é $n_{agua}=1.3330$}. 
Comente a precisão do valor da velocidade de propagação da luz obtida nos 
diferentes meios.
\end{document} 