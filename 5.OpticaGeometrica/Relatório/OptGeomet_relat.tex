%%&program=xelatex
%&encoding=UTF-8 Unicode
% SVN keywords
% $Author: bernardo $
% $Date: 2014-07-24 19:33:49 +0100 (Thu, 24 Jul 2014) $
% $Revision: 6559 $
% $URL: http://metis.ipfn.ist.utl.pt/svn/cdaq/Users/Bernardo/Aulas/LFEB/teXfiles/Planck/Planck.tex $
\documentclass[a4paper,12pt]{article}  % Comments after  % are ignored
%\usepackage{hyperref}                 % For creating hyperlinks in cross references
%
\usepackage{ifxetex}% for XELATEX, or PDFlatex
\usepackage{ifplatform} 
%
\ifxetex
	\usepackage{polyglossia} \setmainlanguage{portuges}
	\usepackage{fontspec}
	\ifwindows
		\setmainfont[Ligatures=TeX]{Garamond}
		\setsansfont[Ligatures=TeX]{Gill Sans MT}
		\setmonofont{Consolas}
%		\setmonofont[Scale=MatchLowercase]{Courier}
	\fi
	\iflinux
		\setmainfont[Ligatures=TeX]{Linux Libertine O}
		\setsansfont[Ligatures=TeX,Scale=MatchLowercase]{Linux Biolinum}
		\setmonofont[Scale=MatchLowercase]{Courier}
	\fi
	\ifmacosx
	% add settings
	% Use xelatex -no-shell ...
	\fi
	\usepackage{xcolor,graphicx} 
\else
	\usepackage[portuguese]{babel}
	%\usepackage[latin1]{inputenc}
	\usepackage[utf8]{inputenc}
	\usepackage[T1]{fontenc}
	\usepackage{graphics}                 % Packages to allow inclusion of graphics
	\usepackage{color}                    % For creating coloured text and background
\fi

\usepackage{enumitem}
\setlist{nolistsep}

\usepackage{amsmath,amssymb,amsfonts} % Typical maths resource packages
\usepackage[retainorgcmds]{IEEEtrantools}
\usepackage{caption}

\usepackage{tikz}
\usetikzlibrary{calc,arrows,decorations.pathmorphing,intersections}

\usepackage[font={small,sf},labelfont={bf},labelsep=endash]{caption}
\usepackage{sansmath}

\def\width{18}
\def\hauteur{11}

\oddsidemargin 0cm
\evensidemargin 0cm

\pagestyle{myheadings}         % Option to put page headers
                               % Needed \documentclass[a4paper,twoside]{article}
\markboth{{MEFT}}
{{\small\it \protect\input{../../LIFE.txt}}}

\addtolength{\hoffset}{-0.5cm}
\addtolength{\textwidth}{2.5cm}
\addtolength{\topmargin}{-1.5cm}
\addtolength{\textheight}{3cm}

%\textwidth 15.5cm
%\topmargin -1.5cm
\setlength{\parindent}{0pt}
\setlength{\parskip}{1ex  plus  0.5ex  minus  0.2ex}
%\parindent 0.5cm
%\textheight 25cm
%\parskip 1mm


% Math macros
\newcommand{\ud}{\,\mathrm{d}} 
\newcommand{\HRule}{\rule{\linewidth}{0.5mm}}

\author{Prof. Bernardo B. Carvalho} 

%%%%, Bernardo Brotas Carvalho\\bernardo.carvalho@tecnico.ulisboa.pt} 
\date{ Setembro 2015} 

\begin{document} 

%	\includegraphics[width=0.2\textwidth]{../logo-ist}%\\[1cm]  %%  Logo_IST_color

%	\HRule \\[0.5cm]
%	{ \huge \sf  \textsc{Construções Geométricas em Lentes Delgadas (aproximação paraxial)} }\\[0.4cm] % \bfseries 
%	{ \large \bfseries Determinação da constante de Planck.}\\

%	{ \large \bfseries Procedimento Experimental}\\
%	\HRule \\%[0.5cm]
%TURNO:________ GRUPO:________ DATA: ___ \\
%Número: _______ Nome: ________________ \\
%Número: _______ Nome: ________________ \\
{  \sf  Relatório da Experiência de Óptica Geométrica} %[0.4cm] % \bfseries 

\input{../../Nomes.txt}

\section{\sf Trabalho preparatório a realizar  ANTES da sessão de Laboratório:}
\begin{enumerate}
\item Descreva por palavras suas quais os objectivos do trabalho que irá realizar na sessão de laboratório.

%\subsubsection{\sf Questões a responder ANTES da sessão de Laboratório:}
%\begin{enumerate}
% (uma folha A4). Indique as expressões que irá utilizar para obter as grandezas experimentais, bem como as expressões para calcular as incertezas. Inclua esta parte também no Relatório. Este irá constituir o ÚNICO meio de consulta na Prova Individual.
%\item A luz emitida pelo díodo vermelho tem um comprimento de onda de $\lambda \approx 600 \, nm$. Calcule $f_{luz}$ e compare com a frequência da modulação da amplitude.
%\item Explique porque razão nesta experiência a luz utilizada  tem de ter a intensidade variável.
%, tal como na experiência da Velocidade do Som.
%\item Uma outra montagem desenvolvida no IST utiliza um díodo laser de longo alcance. No entanto o máximo que se consegue modular a amplitude é, $f_{mod}=1\, MHz$. Calcule a distância
%dos espelhos necessária para se obter o efeito de oposição de fase entre a amplitude dos sinais emitido e o recebido.
\end{enumerate}

\noindent\underline{\makebox[\textwidth][r]{~}} \\
\noindent\underline{\makebox[\textwidth][r]{~}} \\
\noindent\underline{\makebox[\textwidth][r]{~}} \\
\noindent\underline{\makebox[\textwidth][r]{~}} \\
\noindent\underline{\makebox[\textwidth][r]{~}} \\
\noindent\underline{\makebox[\textwidth][r]{~}} \\
\noindent\underline{\makebox[\textwidth][r]{~}} \\

% (uma folha A4). 
%Indique as expressões que irá utilizar para obter as grandezas experimentais, bem como as expressões para calcular as incertezas. Inclua esta parte também no Relatório. Este irá constituir o ÚNICO meio de consulta na Prova Individual.



\subsubsection{\sf Equações }
Escreva no seguinte quadro todas as equações necessárias para calcular as grandezas, bem como as suas incertezas.
\begin{center}
\framebox[15cm]{\rule{0pt}{8.5cm}}
\end{center}

\newpage
\section{\sf Relatório}
\subsection{\sf Montagem Experimental}
Desenhe um diagrama das diversas montagens experimentais que realizou. Inclua em anexo os esquemas de traçado de raios em papel milimétrico.

\begin{center}
\framebox[18cm]{\rule{0pt}{10cm}}
\end{center}

%%%%%%%%%%%%%%%%%%%%%%%%%%%%%%%%%%%%%%%%%%%%%%%%%

\subsection{\sf Cálculo do índice de refracção de um vidro acrílico}
Preencha as seguintes tabelas indicando  apenas os algarismos significativos. Terá que verificar as contas com auxílio da calculadora, para um dos ensaios e na presença do docente. \emph{Todos os ângulos deverão ser indicados em graus}. 

\newpage
\subsubsection{\sf Face plana}

$\epsilon_{\theta_i}=$~\underline{\makebox[1cm][r]{~}} $^\circ$; 
$\epsilon_{\theta_t}=$~\underline{\makebox[1cm][r]{~}} $^\circ$;
$\epsilon_{\theta_r}=$~\underline{\makebox[1cm][r]{~}} $^\circ$

\begin{center}
	%\centering
	\begin{tabular}{|c|c|c|c|c|c|c|c|c|}
	\hline
	 Ensaio   &  $\theta_i $  & $\sin\theta_i$  &  $\theta_r $ esq.  & $\theta_t $ esq.  & $\theta_r $ dir.  & $\theta_t $ dir. & $\overline{\theta_t }$  & $\sin\overline{\theta_t}$ \\
	\hline \hline
	 1 & $\qquad$ & $\qquad\pm\qquad$ & & & & & $\qquad\pm\qquad$ & $\qquad\pm\qquad$   \\\cline{1-9}
	 2 & & $\quad\pm\quad$ & & & & & $\quad\pm\quad$ & $\quad\pm\quad$   \\\cline{1-9}
	 3 & & $\quad\pm\quad$ & & & & & $\quad\pm\quad$ & $\quad\pm\quad$   \\\cline{1-9}
	 4 & & $\quad\pm\quad$ & & & & & $\quad\pm\quad$ & $\quad\pm\quad$   \\\cline{1-9}
	 5 & & $\quad\pm\quad$ & & & & & $\quad\pm\quad$ & $\quad\pm\quad$   \\\cline{1-9}
	 6 & & $\quad\pm\quad$ & & & & & $\quad\pm\quad$ & $\quad\pm\quad$   \\\cline{1-9}
	 7 & & $\quad\pm\quad$ & & & & & $\quad\pm\quad$ & $\quad\pm\quad$   \\\cline{1-9}
	 8 & & $\quad\pm\quad$ & & & & & $\quad\pm\quad$ & $\quad\pm\quad$   \\\cline{1-9}
	 9 & & $\quad\pm\quad$ & & & & & $\quad\pm\quad$ & $\quad\pm\quad$   \\
	  \hline
	\end{tabular}
\end{center}



%\noindent Valor médio: $\overline{n}_{vidro} =$~\underline{\makebox[2cm][r]{~}}$\pm$~\underline{\makebox[2cm][r]{~}}

Valor obtido pelo gráfico: $\overline{n}_{vidro} =$~\underline{\makebox[2cm][r]{~}}$\pm$~\underline{\makebox[2cm][r]{~}}


%%%%%%%%%%%%%%%%%%%%%%%%%%%%%%%%%%%%%%%%%%%%%%%%%


\subsubsection{\sf Face cilíndrica}


$\epsilon_{\theta_i}=$~\underline{\makebox[1cm][r]{~}} $^\circ$; 
$\epsilon_{\theta_t}=$~\underline{\makebox[1cm][r]{~}} $^\circ$;
$\epsilon_{\theta_r}=$~\underline{\makebox[1cm][r]{~}} $^\circ$


\begin{center}
	%\centering
	\begin{tabular}{|c|c|c|c|c|c|c|c|c|}
	\hline
	 Ensaio   &  $\theta_i $  & $\sin\theta_i$  &  $\theta_r $ esq.  & $\theta_t $ esq.  & $\theta_r $ dir.  & $\theta_t $ dir. & $\overline{\theta_t }$  & $\sin\overline{\theta_t}$ \\
	\hline \hline
	 1 & $\qquad$ & $\qquad\pm\qquad$ & & & & & $\qquad\pm\qquad$ & $\qquad\pm\qquad$   \\\cline{1-9}
	 2 & & $\quad\pm\quad$ & & & & & $\quad\pm\quad$ & $\quad\pm\quad$   \\\cline{1-9}
	 3 & & $\quad\pm\quad$ & & & & & $\quad\pm\quad$ & $\quad\pm\quad$   \\\cline{1-9}
	 4 & & $\quad\pm\quad$ & & & & & $\quad\pm\quad$ & $\quad\pm\quad$   \\\cline{1-9}
	 5 & & $\quad\pm\quad$ & & & & & $\quad\pm\quad$ & $\quad\pm\quad$   \\\cline{1-9}
	 6 & & $\quad\pm\quad$ & & & & & $\quad\pm\quad$ & $\quad\pm\quad$   \\\cline{1-9}
	 7 & & $\quad\pm\quad$ & & & & & $\quad\pm\quad$ & $\quad\pm\quad$   \\\cline{1-9}
	 8 & & $\quad\pm\quad$ & & & & & $\quad\pm\quad$ & $\quad\pm\quad$   \\\cline{1-9}
	 9 & & $\quad\pm\quad$ & & & & & $\quad\pm\quad$ & $\quad\pm\quad$   \\
	  \hline
	\end{tabular}
\end{center}

Valor obtido pelo gráfico: $\overline{n}_{vidro} =$~\underline{\makebox[2cm][r]{~}}$\pm$~\underline{\makebox[2cm][r]{~}}

%%%%%%%%%%%%%%%%%%%%%%%%%%%%%%%%%%%%%%%%%%%%%%%%%

\subsubsection{\sf Ângulo-limite}

\begin{center}
	%\centering
	\begin{tabular}{|c|c|c|c|}
	\hline
	 Ensaio   &  $\theta_{lim}$ esq.  & $\theta_{lim}$ dir.  &  $\overline{\theta}_{lim}$  \\
	\hline \hline
	 1 & & & $\qquad\pm\qquad$ \\\cline{1-4}
	 2 & & & $\qquad\pm\qquad$   \\\cline{1-4}
	 3 & & & $\qquad\pm\qquad$   \\
	  \hline
	\end{tabular}
\end{center}


%\noindent Valor médio: $\overline{n}_{vidro} =$~\underline{\makebox[2cm][r]{~}}$\pm$~\underline{\makebox[2cm][r]{~}}

Ângulo limite: $\overline{\theta}_{lim} =$~\underline{\makebox[2cm][r]{~}}$\pm$~\underline{\makebox[2cm][r]{~}}

Valor obtido pelo ângulo limite: $\overline{n}_{vidro} =$~\underline{\makebox[2cm][r]{~}}$\pm$~\underline{\makebox[2cm][r]{~}}



%%%%%%%%%%%%%%%%%%%%%%%%%%%%%%%%%%%%%%%%%%%%%%%%%

\subsection{\sf Polarização da luz -- ângulo de Brewster}\label{sec:dados_ar}

Para o cálculo do desvio à exatidão, considere como exato o valor $n=1,50$.

Ângulo de Brewster calculado: $\theta_B=~\underline{\makebox[2cm][r]{~}}$



\begin{center}
	%\centering
	\begin{tabular}{|c|c|c|c|c|}
	\hline \hline
	 Ensaio & $\theta_{Bmin} $ & $\theta_{Bmax} $ & $\overline{\theta_{B}} $  & Desv. Exatidão \\
	 \hline \hline
	 1 & & & $\qquad\pm\qquad$ & \\ \cline{1-5}
	 2 & & & $\qquad\pm\qquad$ & \\ \cline{1-5}
	 3 & & & $\qquad\pm\qquad$  & \\ \hline
			\end{tabular}
\end{center}

%%%%%%%%%%%%%%%%%%%%%%%%%%%%%%%%%%%%%%%


\subsection{\sf Distância focal de uma lente convergente}



\subsubsection{\sf Método direto}

Distância entre lente colimadora e fonte luminosa: $~\underline{\makebox[2cm][r]{~}}$ mm

\begin{center}
	\begin{tabular}{|c|c|c|c|}
	\hline
	 Ensaio & $ f_{min}$ (mm) & $ f_{max}$ (mm)  & $\overline{f}$ (mm)\\
	\hline \hline
	 1 & \makebox[2cm][r]{} &  \makebox[2cm][r]{}  & \makebox[2cm][r]{}  \\ \cline{1-3}
	  2 & & &  $\quad \pm \quad$ \\ \cline{1-3}
  	  3 & & & \\  \hline
			\end{tabular}
\end{center}
%%%%%%%%%%%%%%%%%%%%%%%%%%%%%%%%%%%%%%%
\subsubsection{\sf Método da equação dos focos conjugados e ampliação}

1) Distância entre lente convergente e objecto: $D_O=~\underline{\makebox[1cm][r]{~}}\pm~\underline{\makebox[1cm][r]{~}}$ mm

\begin{center}
	\begin{tabular}{|c|c|c|c|c|c|c|}
	\hline
	 Ensaio &
	  $ D_{I}^{min}$ (mm) & 
	  $ D_I^{max}$ (mm) & 
	  $ \overline{D_{I}}\pm\epsilon_{D_i}$ (mm) &
	  $A=\overline{D_I}/D_O$ &
	  $f$ (mm) \\
	\hline \hline
	1 & & &  $\quad \pm \quad$ & $\quad \pm \quad$& $\qquad \pm \qquad$ \\ \cline{1-6}
	2 & & & $\quad \pm \quad$ & $\quad \pm \quad$& $\quad \pm \quad$  \\ \cline{1-6}
	3 & & & $\quad \pm \quad$ & $\quad \pm \quad$& $\quad \pm \quad$ \\ \hline
	  \end{tabular}
\end{center}
 $\overline{f} =$ ~\underline{\makebox[1cm][r]{~}}$\pm$~\underline{\makebox[1cm][r]{~}} (mm);  $\overline{A} =$ ~\underline{\makebox[1cm][r]{~}}$\pm$~\underline{\makebox[1cm][r]{~}}

\begin{center}
	\begin{tabular}{|c|c|c|c|c|c|c|}
	\hline
	 Ensaio &
	  $h_O$ (mm) & 
	  $\epsilon_{h_O}$ (mm) & 
	  $h_I$ (mm) & 
	  $\epsilon_{h_I}$ (mm) & 
	  Ampliação $A$ &
  	  $\overline{A}$\\
	\hline \hline
	1 & & & & & $\quad \pm \quad$  & \makebox[2cm][r]{}  \\ \cline{1-6}
	2 & & & & & $\quad \pm \quad$ & $\quad \pm \quad$\\ \cline{1-6}
	3 & & & & & $\quad \pm \quad$ & \\  \hline
			\end{tabular}
\end{center}

2) Distância entre lente convergente e objecto: $D_O=~\underline{\makebox[1cm][r]{~}}\pm~\underline{\makebox[1cm][r]{~}}$ mm

\begin{center}
	\begin{tabular}{|c|c|c|c|c|c|c|}
	\hline
	 Ensaio &
	  $ D_{I}^{min}$ (mm) & 
	  $ D_I^{max}$ (mm) & 
	  $ \overline{D_{I}}\pm\epsilon_{D_i}$ (mm) &
	  $A=\overline{D_I}/D_O$ &
	  $f$ (mm) \\
	\hline \hline
	1 & & &  $\quad \pm \quad$ & $\quad \pm \quad$& $\qquad \pm \qquad$ \\ \cline{1-6}
	2 & & & $\quad \pm \quad$ & $\quad \pm \quad$& $\quad \pm \quad$  \\ \cline{1-6}
	3 & & & $\quad \pm \quad$ & $\quad \pm \quad$& $\quad \pm \quad$ \\ \hline
	  \end{tabular}
\end{center}
 $\overline{f} =$ ~\underline{\makebox[1cm][r]{~}}$\pm$~\underline{\makebox[1cm][r]{~}} (mm);  $\overline{A} =$ ~\underline{\makebox[1cm][r]{~}}$\pm$~\underline{\makebox[1cm][r]{~}}

\begin{center}
	\begin{tabular}{|c|c|c|c|c|c|c|}
	\hline
	 Ensaio &
	  $h_O$ (mm) & 
	  $\epsilon_{h_O}$ (mm) & 
	  $h_I$ (mm) & 
	  $\epsilon_{h_I}$ (mm) & 
	  Ampliação $A$ &
  	  $\overline{A}$\\
	\hline \hline
	1 & & & & & $\quad \pm \quad$  & \makebox[2cm][r]{}  \\ \cline{1-6}
	2 & & & & & $\quad \pm \quad$ & $\quad \pm \quad$\\ \cline{1-6}
	3 & & & & & $\quad \pm \quad$ & \\  \hline
			\end{tabular}
\end{center}

Analise e comente os resultados que obteve usando o método directo e o método dos focos conjugados.

\noindent\underline{\makebox[\textwidth][r]{~}} \\
\noindent\underline{\makebox[\textwidth][r]{~}} \\
\noindent\underline{\makebox[\textwidth][r]{~}} \\
\noindent\underline{\makebox[\textwidth][r]{~}} \\

%%%%%%%%%%%%%%%%%%%%%%%%%%%%%%%%%%%%%%%

\subsection{\sf Distância focal de uma lente divergente}

Distância entre lentes e objecto: $D_O=~\underline{\makebox[1cm][r]{~}}\pm~\underline{\makebox[1cm][r]{~}}$ mm

\begin{center}
	\begin{tabular}{|c|c|c|c|c|c|c|}
	\hline
	 Ensaio &
	  $ D_{I}^{min}$ (mm) & 
	  $ D_I^{max}$ (mm) & 
	  $ \overline{D_{I}}\pm\epsilon_{D_i}$ (mm) &
	  $f\pm\epsilon_f$ (mm)\\
	\hline \hline
	1 & &  $\quad \pm \quad$ & $\quad \pm \quad$& $\qquad \pm \qquad$ \\ \cline{1-5}
	2 & & $\quad \pm \quad$ & $\quad \pm \quad$& $\quad \pm \quad$ \\ \hline
	  \end{tabular}
\end{center}
 $\overline{f} =$ ~\underline{\makebox[1cm][r]{~}}$\pm$~\underline{\makebox[1cm][r]{~}} (mm); 



%%%%%%%%%%%%%%%%%%%%%%%%%%%%%%%%%%%%%%%
\subsection{\sf Análise, Conclusões e Comentários}
\noindent\underline{\makebox[\textwidth][r]{~}} \\
\noindent\underline{\makebox[\textwidth][r]{~}} \\
\noindent\underline{\makebox[\textwidth][r]{~}} \\
\noindent\underline{\makebox[\textwidth][r]{~}} \\
\noindent\underline{\makebox[\textwidth][r]{~}} \\
\noindent\underline{\makebox[\textwidth][r]{~}} \\
\noindent\underline{\makebox[\textwidth][r]{~}} \\
\noindent\underline{\makebox[\textwidth][r]{~}} \\
\noindent\underline{\makebox[\textwidth][r]{~}} \\
\noindent\underline{\makebox[\textwidth][r]{~}} \\
\noindent\underline{\makebox[\textwidth][r]{~}} \\
\noindent\underline{\makebox[\textwidth][r]{~}} \\
\noindent\underline{\makebox[\textwidth][r]{~}} \\
\noindent\underline{\makebox[\textwidth][r]{~}} \\
\noindent\underline{\makebox[\textwidth][r]{~}} \\
\noindent\underline{\makebox[\textwidth][r]{~}} \\
\noindent\underline{\makebox[\textwidth][r]{~}} \\



%\newpage

\end{document} 